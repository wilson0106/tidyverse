\PassOptionsToPackage{unicode=true}{hyperref} % options for packages loaded elsewhere
\PassOptionsToPackage{hyphens}{url}
%
\documentclass[]{book}
\usepackage{lmodern}
\usepackage{amssymb,amsmath}
\usepackage{ifxetex,ifluatex}
\usepackage{fixltx2e} % provides \textsubscript
\ifnum 0\ifxetex 1\fi\ifluatex 1\fi=0 % if pdftex
  \usepackage[T1]{fontenc}
  \usepackage[utf8]{inputenc}
  \usepackage{textcomp} % provides euro and other symbols
\else % if luatex or xelatex
  \usepackage{unicode-math}
  \defaultfontfeatures{Ligatures=TeX,Scale=MatchLowercase}
\fi
% use upquote if available, for straight quotes in verbatim environments
\IfFileExists{upquote.sty}{\usepackage{upquote}}{}
% use microtype if available
\IfFileExists{microtype.sty}{%
\usepackage[]{microtype}
\UseMicrotypeSet[protrusion]{basicmath} % disable protrusion for tt fonts
}{}
\IfFileExists{parskip.sty}{%
\usepackage{parskip}
}{% else
\setlength{\parindent}{0pt}
\setlength{\parskip}{6pt plus 2pt minus 1pt}
}
\usepackage{hyperref}
\hypersetup{
            pdftitle={Tidyverse - aula 1 - Introdução},
            pdfauthor={Wilson Souza},
            pdfborder={0 0 0},
            breaklinks=true}
\urlstyle{same}  % don't use monospace font for urls
\usepackage{color}
\usepackage{fancyvrb}
\newcommand{\VerbBar}{|}
\newcommand{\VERB}{\Verb[commandchars=\\\{\}]}
\DefineVerbatimEnvironment{Highlighting}{Verbatim}{commandchars=\\\{\}}
% Add ',fontsize=\small' for more characters per line
\usepackage{framed}
\definecolor{shadecolor}{RGB}{248,248,248}
\newenvironment{Shaded}{\begin{snugshade}}{\end{snugshade}}
\newcommand{\AlertTok}[1]{\textcolor[rgb]{0.94,0.16,0.16}{#1}}
\newcommand{\AnnotationTok}[1]{\textcolor[rgb]{0.56,0.35,0.01}{\textbf{\textit{#1}}}}
\newcommand{\AttributeTok}[1]{\textcolor[rgb]{0.77,0.63,0.00}{#1}}
\newcommand{\BaseNTok}[1]{\textcolor[rgb]{0.00,0.00,0.81}{#1}}
\newcommand{\BuiltInTok}[1]{#1}
\newcommand{\CharTok}[1]{\textcolor[rgb]{0.31,0.60,0.02}{#1}}
\newcommand{\CommentTok}[1]{\textcolor[rgb]{0.56,0.35,0.01}{\textit{#1}}}
\newcommand{\CommentVarTok}[1]{\textcolor[rgb]{0.56,0.35,0.01}{\textbf{\textit{#1}}}}
\newcommand{\ConstantTok}[1]{\textcolor[rgb]{0.00,0.00,0.00}{#1}}
\newcommand{\ControlFlowTok}[1]{\textcolor[rgb]{0.13,0.29,0.53}{\textbf{#1}}}
\newcommand{\DataTypeTok}[1]{\textcolor[rgb]{0.13,0.29,0.53}{#1}}
\newcommand{\DecValTok}[1]{\textcolor[rgb]{0.00,0.00,0.81}{#1}}
\newcommand{\DocumentationTok}[1]{\textcolor[rgb]{0.56,0.35,0.01}{\textbf{\textit{#1}}}}
\newcommand{\ErrorTok}[1]{\textcolor[rgb]{0.64,0.00,0.00}{\textbf{#1}}}
\newcommand{\ExtensionTok}[1]{#1}
\newcommand{\FloatTok}[1]{\textcolor[rgb]{0.00,0.00,0.81}{#1}}
\newcommand{\FunctionTok}[1]{\textcolor[rgb]{0.00,0.00,0.00}{#1}}
\newcommand{\ImportTok}[1]{#1}
\newcommand{\InformationTok}[1]{\textcolor[rgb]{0.56,0.35,0.01}{\textbf{\textit{#1}}}}
\newcommand{\KeywordTok}[1]{\textcolor[rgb]{0.13,0.29,0.53}{\textbf{#1}}}
\newcommand{\NormalTok}[1]{#1}
\newcommand{\OperatorTok}[1]{\textcolor[rgb]{0.81,0.36,0.00}{\textbf{#1}}}
\newcommand{\OtherTok}[1]{\textcolor[rgb]{0.56,0.35,0.01}{#1}}
\newcommand{\PreprocessorTok}[1]{\textcolor[rgb]{0.56,0.35,0.01}{\textit{#1}}}
\newcommand{\RegionMarkerTok}[1]{#1}
\newcommand{\SpecialCharTok}[1]{\textcolor[rgb]{0.00,0.00,0.00}{#1}}
\newcommand{\SpecialStringTok}[1]{\textcolor[rgb]{0.31,0.60,0.02}{#1}}
\newcommand{\StringTok}[1]{\textcolor[rgb]{0.31,0.60,0.02}{#1}}
\newcommand{\VariableTok}[1]{\textcolor[rgb]{0.00,0.00,0.00}{#1}}
\newcommand{\VerbatimStringTok}[1]{\textcolor[rgb]{0.31,0.60,0.02}{#1}}
\newcommand{\WarningTok}[1]{\textcolor[rgb]{0.56,0.35,0.01}{\textbf{\textit{#1}}}}
\usepackage{longtable,booktabs}
% Fix footnotes in tables (requires footnote package)
\IfFileExists{footnote.sty}{\usepackage{footnote}\makesavenoteenv{longtable}}{}
\usepackage{graphicx,grffile}
\makeatletter
\def\maxwidth{\ifdim\Gin@nat@width>\linewidth\linewidth\else\Gin@nat@width\fi}
\def\maxheight{\ifdim\Gin@nat@height>\textheight\textheight\else\Gin@nat@height\fi}
\makeatother
% Scale images if necessary, so that they will not overflow the page
% margins by default, and it is still possible to overwrite the defaults
% using explicit options in \includegraphics[width, height, ...]{}
\setkeys{Gin}{width=\maxwidth,height=\maxheight,keepaspectratio}
\setlength{\emergencystretch}{3em}  % prevent overfull lines
\providecommand{\tightlist}{%
  \setlength{\itemsep}{0pt}\setlength{\parskip}{0pt}}
\setcounter{secnumdepth}{5}
% Redefines (sub)paragraphs to behave more like sections
\ifx\paragraph\undefined\else
\let\oldparagraph\paragraph
\renewcommand{\paragraph}[1]{\oldparagraph{#1}\mbox{}}
\fi
\ifx\subparagraph\undefined\else
\let\oldsubparagraph\subparagraph
\renewcommand{\subparagraph}[1]{\oldsubparagraph{#1}\mbox{}}
\fi

% set default figure placement to htbp
\makeatletter
\def\fps@figure{htbp}
\makeatother

\usepackage{booktabs}
\usepackage[]{natbib}
\bibliographystyle{apalike}

\title{Tidyverse - aula 1 - Introdução}
\author{Wilson Souza}
\date{2021-03-17}

\begin{document}
\maketitle

{
\setcounter{tocdepth}{1}
\tableofcontents
}
\hypertarget{introduuxe7uxe3o---pacotes}{%
\chapter{Introdução - Pacotes}\label{introduuxe7uxe3o---pacotes}}

Para começar precisamos instalar os seguintes pacotes:

Ou então, um único pacote \textbf{tidyverse}. Que engloba todos os pacotes acima.

Após instalados precisamos carrega-los. Podemos fazer isso chamando cada um dos pacotes separadamente.

\begin{Shaded}
\begin{Highlighting}[]
\KeywordTok{library}\NormalTok{(tibble)}
\KeywordTok{library}\NormalTok{(dplyr)}
\end{Highlighting}
\end{Shaded}

\begin{verbatim}
## 
## Attaching package: 'dplyr'
\end{verbatim}

\begin{verbatim}
## The following objects are masked from 'package:stats':
## 
##     filter, lag
\end{verbatim}

\begin{verbatim}
## The following objects are masked from 'package:base':
## 
##     intersect, setdiff, setequal, union
\end{verbatim}

\begin{Shaded}
\begin{Highlighting}[]
\KeywordTok{library}\NormalTok{(tidyr)}
\KeywordTok{library}\NormalTok{(magrittr) }\CommentTok{# não faz parte do pacote tidyverse, mas é carregado por ele, dada a importância do operador pipe.}
\end{Highlighting}
\end{Shaded}

\begin{verbatim}
## 
## Attaching package: 'magrittr'
\end{verbatim}

\begin{verbatim}
## The following object is masked from 'package:tidyr':
## 
##     extract
\end{verbatim}

Ou então podemos chamar apenas o pacote \emph{tidyverse}.

\begin{Shaded}
\begin{Highlighting}[]
\KeywordTok{library}\NormalTok{(tidyverse)}
\end{Highlighting}
\end{Shaded}

\begin{verbatim}
## -- Attaching packages --------------------------------------- tidyverse 1.3.0 --
\end{verbatim}

\begin{verbatim}
## v ggplot2 3.3.3     v stringr 1.4.0
## v readr   1.4.0     v forcats 0.5.1
## v purrr   0.3.4
\end{verbatim}

\begin{verbatim}
## -- Conflicts ------------------------------------------ tidyverse_conflicts() --
## x magrittr::extract() masks tidyr::extract()
## x dplyr::filter()     masks stats::filter()
## x dplyr::lag()        masks stats::lag()
## x purrr::set_names()  masks magrittr::set_names()
\end{verbatim}

Vamos trabalhar \textbf{algumas} funções destes pacotes.

\hypertarget{do-pacote-tibble-iremos-trabalhar-as-seguintes-funuxe7uxf5es}{%
\subsection{\texorpdfstring{Do pacote \emph{tibble} iremos trabalhar as seguintes funções:}{Do pacote tibble iremos trabalhar as seguintes funções:}}\label{do-pacote-tibble-iremos-trabalhar-as-seguintes-funuxe7uxf5es}}

\begin{itemize}
\tightlist
\item
  add\_column()
\item
  add\_row()
\item
  as\_tibble()
\item
  column\_to\_rownames()
\item
  remove\_rownames()
\item
  rownames\_to\_column()
\item
  subsetting
\end{itemize}

\hypertarget{do-pacote-dplyr-iremos-trabalhas-as-seguintes-funuxe7uxf5es}{%
\subsection{\texorpdfstring{Do pacote \emph{dplyr} iremos trabalhas as seguintes funções:}{Do pacote dplyr iremos trabalhas as seguintes funções:}}\label{do-pacote-dplyr-iremos-trabalhas-as-seguintes-funuxe7uxf5es}}

\begin{itemize}
\tightlist
\item
  add\_count()
\item
  arrange()
\item
  bind\_cols()
\item
  bind\_rows()
\item
  count()
\item
  desc()
\item
  distinct()
\item
  filter()
\item
  group\_by()
\item
  mutate()
\item
  na\_if()
\item
  recode\_factor()
\item
  relocate()
\item
  rename()
\item
  select()
\item
  slice()
\item
  summarise()
\item
  transmute()
\end{itemize}

\hypertarget{do-pacote-tidyr-iremos-trabalhar-as-seguintes-funuxe7uxf5es}{%
\subsection{\texorpdfstring{Do pacote \emph{tidyr} iremos trabalhar as seguintes funções:}{Do pacote tidyr iremos trabalhar as seguintes funções:}}\label{do-pacote-tidyr-iremos-trabalhar-as-seguintes-funuxe7uxf5es}}

\begin{itemize}
\tightlist
\item
  drop\_na()
\item
  fill()
\item
  pivot\_longer()
\item
  pivot\_wider()
\item
  replace\_na()
\item
  separate()
\item
  unite()
\end{itemize}

\hypertarget{do-pacote-magrittr-iremos-trabalhar-com-o-operador-chamado-pipe}{%
\subsection{\texorpdfstring{Do pacote \emph{magrittr} iremos trabalhar com o operador chamado \textbf{pipe}:}{Do pacote magrittr iremos trabalhar com o operador chamado pipe:}}\label{do-pacote-magrittr-iremos-trabalhar-com-o-operador-chamado-pipe}}

\begin{itemize}
\tightlist
\item
  \%\textgreater{}\%\footnote{Vamos utilizar este operador em todos os exemplos}
\end{itemize}

\hypertarget{importando-os-dados}{%
\section{Importando os dados}\label{importando-os-dados}}

Caso não souberem importar os dados de um arquivo *.csv ou *.excel presente no seu PC ao fim da aula eu explico. Os exemplos de como as funções trabalham seguem do mesmo jeito.

Vamos trabalhar com conjuntos de dados presentes no R.

\begin{Shaded}
\begin{Highlighting}[]
\KeywordTok{data}\NormalTok{(mtcars)}
\NormalTok{carros <-}\StringTok{ }\KeywordTok{as_tibble}\NormalTok{(mtcars, }\DataTypeTok{rownames =} \OtherTok{NA}\NormalTok{)}

\KeywordTok{data}\NormalTok{(iris)}
\NormalTok{flores <-}\StringTok{ }\KeywordTok{as_tibble}\NormalTok{(iris)}

\KeywordTok{data}\NormalTok{(starwars)}
\NormalTok{starwars <-}\StringTok{ }\KeywordTok{as_tibble}\NormalTok{(starwars)}
\end{Highlighting}
\end{Shaded}

\hypertarget{tibble}{%
\chapter{\texorpdfstring{\emph{tibble}}{tibble}}\label{tibble}}

\hypertarget{as_tibble---converte-um-objeto-do-tipo-matriz-ou-data-frame-para-tbl_df}{%
\section{as\_tibble() - converte um objeto do tipo matriz ou data frame para tbl\_df}\label{as_tibble---converte-um-objeto-do-tipo-matriz-ou-data-frame-para-tbl_df}}

\begin{Shaded}
\begin{Highlighting}[]
\NormalTok{obj1 <-}\StringTok{ }\NormalTok{carros}
\NormalTok{obj2 <-}\StringTok{ }\NormalTok{carros }\OperatorTok
\StringTok{  }\KeywordTok{data.frame}\NormalTok{()}

\CommentTok{# ["linhas", "colunas"]}
\NormalTok{obj1[, }\DecValTok{1}\NormalTok{]}
\end{Highlighting}
\end{Shaded}

\begin{verbatim}
## # A tibble: 32 x 1
##      mpg
##    <dbl>
##  1  21  
##  2  21  
##  3  22.8
##  4  21.4
##  5  18.7
##  6  18.1
##  7  14.3
##  8  24.4
##  9  22.8
## 10  19.2
## # ... with 22 more rows
\end{verbatim}

\begin{Shaded}
\begin{Highlighting}[]
\NormalTok{obj2[, }\DecValTok{1}\NormalTok{]}
\end{Highlighting}
\end{Shaded}

\begin{verbatim}
##  [1] 21.0 21.0 22.8 21.4 18.7 18.1 14.3 24.4 22.8 19.2 17.8 16.4 17.3 15.2 10.4
## [16] 10.4 14.7 32.4 30.4 33.9 21.5 15.5 15.2 13.3 19.2 27.3 26.0 30.4 15.8 19.7
## [31] 15.0 21.4
\end{verbatim}

\hypertarget{add_column---adiciona-uma-coluna-nova-a-planilha}{%
\section{add\_column() - adiciona uma coluna nova a planilha}\label{add_column---adiciona-uma-coluna-nova-a-planilha}}

\begin{Shaded}
\begin{Highlighting}[]
\NormalTok{carros }\OperatorTok
\StringTok{  }\KeywordTok{add_column}\NormalTok{(}\DataTypeTok{teste =} \DecValTok{0}\NormalTok{)}
\end{Highlighting}
\end{Shaded}

\begin{verbatim}
## # A tibble: 32 x 12
##      mpg   cyl  disp    hp  drat    wt  qsec    vs    am  gear  carb teste
##    <dbl> <dbl> <dbl> <dbl> <dbl> <dbl> <dbl> <dbl> <dbl> <dbl> <dbl> <dbl>
##  1  21       6  160    110  3.9   2.62  16.5     0     1     4     4     0
##  2  21       6  160    110  3.9   2.88  17.0     0     1     4     4     0
##  3  22.8     4  108     93  3.85  2.32  18.6     1     1     4     1     0
##  4  21.4     6  258    110  3.08  3.22  19.4     1     0     3     1     0
##  5  18.7     8  360    175  3.15  3.44  17.0     0     0     3     2     0
##  6  18.1     6  225    105  2.76  3.46  20.2     1     0     3     1     0
##  7  14.3     8  360    245  3.21  3.57  15.8     0     0     3     4     0
##  8  24.4     4  147.    62  3.69  3.19  20       1     0     4     2     0
##  9  22.8     4  141.    95  3.92  3.15  22.9     1     0     4     2     0
## 10  19.2     6  168.   123  3.92  3.44  18.3     1     0     4     4     0
## # ... with 22 more rows
\end{verbatim}

\begin{Shaded}
\begin{Highlighting}[]
\NormalTok{carros}
\end{Highlighting}
\end{Shaded}

\begin{verbatim}
## # A tibble: 32 x 11
##      mpg   cyl  disp    hp  drat    wt  qsec    vs    am  gear  carb
##  * <dbl> <dbl> <dbl> <dbl> <dbl> <dbl> <dbl> <dbl> <dbl> <dbl> <dbl>
##  1  21       6  160    110  3.9   2.62  16.5     0     1     4     4
##  2  21       6  160    110  3.9   2.88  17.0     0     1     4     4
##  3  22.8     4  108     93  3.85  2.32  18.6     1     1     4     1
##  4  21.4     6  258    110  3.08  3.22  19.4     1     0     3     1
##  5  18.7     8  360    175  3.15  3.44  17.0     0     0     3     2
##  6  18.1     6  225    105  2.76  3.46  20.2     1     0     3     1
##  7  14.3     8  360    245  3.21  3.57  15.8     0     0     3     4
##  8  24.4     4  147.    62  3.69  3.19  20       1     0     4     2
##  9  22.8     4  141.    95  3.92  3.15  22.9     1     0     4     2
## 10  19.2     6  168.   123  3.92  3.44  18.3     1     0     4     4
## # ... with 22 more rows
\end{verbatim}

\begin{Shaded}
\begin{Highlighting}[]
\NormalTok{carros }\OperatorTok\StringTok{ }
\StringTok{  }\KeywordTok{add_column}\NormalTok{(}\DataTypeTok{teste =} \DecValTok{0}\NormalTok{, }\DataTypeTok{.before =} \StringTok{"mpg"}\NormalTok{)}
\end{Highlighting}
\end{Shaded}

\begin{verbatim}
## # A tibble: 32 x 12
##    teste   mpg   cyl  disp    hp  drat    wt  qsec    vs    am  gear  carb
##    <dbl> <dbl> <dbl> <dbl> <dbl> <dbl> <dbl> <dbl> <dbl> <dbl> <dbl> <dbl>
##  1     0  21       6  160    110  3.9   2.62  16.5     0     1     4     4
##  2     0  21       6  160    110  3.9   2.88  17.0     0     1     4     4
##  3     0  22.8     4  108     93  3.85  2.32  18.6     1     1     4     1
##  4     0  21.4     6  258    110  3.08  3.22  19.4     1     0     3     1
##  5     0  18.7     8  360    175  3.15  3.44  17.0     0     0     3     2
##  6     0  18.1     6  225    105  2.76  3.46  20.2     1     0     3     1
##  7     0  14.3     8  360    245  3.21  3.57  15.8     0     0     3     4
##  8     0  24.4     4  147.    62  3.69  3.19  20       1     0     4     2
##  9     0  22.8     4  141.    95  3.92  3.15  22.9     1     0     4     2
## 10     0  19.2     6  168.   123  3.92  3.44  18.3     1     0     4     4
## # ... with 22 more rows
\end{verbatim}

\hypertarget{add_row---adiciona-uma-nova-linha-a-planilha}{%
\section{add\_row() - adiciona uma nova linha a planilha}\label{add_row---adiciona-uma-nova-linha-a-planilha}}

\begin{Shaded}
\begin{Highlighting}[]
\NormalTok{carros }\OperatorTok
\StringTok{  }\KeywordTok{add_row}\NormalTok{()}
\end{Highlighting}
\end{Shaded}

\begin{verbatim}
## # A tibble: 33 x 11
##      mpg   cyl  disp    hp  drat    wt  qsec    vs    am  gear  carb
##  * <dbl> <dbl> <dbl> <dbl> <dbl> <dbl> <dbl> <dbl> <dbl> <dbl> <dbl>
##  1  21       6  160    110  3.9   2.62  16.5     0     1     4     4
##  2  21       6  160    110  3.9   2.88  17.0     0     1     4     4
##  3  22.8     4  108     93  3.85  2.32  18.6     1     1     4     1
##  4  21.4     6  258    110  3.08  3.22  19.4     1     0     3     1
##  5  18.7     8  360    175  3.15  3.44  17.0     0     0     3     2
##  6  18.1     6  225    105  2.76  3.46  20.2     1     0     3     1
##  7  14.3     8  360    245  3.21  3.57  15.8     0     0     3     4
##  8  24.4     4  147.    62  3.69  3.19  20       1     0     4     2
##  9  22.8     4  141.    95  3.92  3.15  22.9     1     0     4     2
## 10  19.2     6  168.   123  3.92  3.44  18.3     1     0     4     4
## # ... with 23 more rows
\end{verbatim}

\begin{Shaded}
\begin{Highlighting}[]
\NormalTok{carros}
\end{Highlighting}
\end{Shaded}

\begin{verbatim}
## # A tibble: 32 x 11
##      mpg   cyl  disp    hp  drat    wt  qsec    vs    am  gear  carb
##  * <dbl> <dbl> <dbl> <dbl> <dbl> <dbl> <dbl> <dbl> <dbl> <dbl> <dbl>
##  1  21       6  160    110  3.9   2.62  16.5     0     1     4     4
##  2  21       6  160    110  3.9   2.88  17.0     0     1     4     4
##  3  22.8     4  108     93  3.85  2.32  18.6     1     1     4     1
##  4  21.4     6  258    110  3.08  3.22  19.4     1     0     3     1
##  5  18.7     8  360    175  3.15  3.44  17.0     0     0     3     2
##  6  18.1     6  225    105  2.76  3.46  20.2     1     0     3     1
##  7  14.3     8  360    245  3.21  3.57  15.8     0     0     3     4
##  8  24.4     4  147.    62  3.69  3.19  20       1     0     4     2
##  9  22.8     4  141.    95  3.92  3.15  22.9     1     0     4     2
## 10  19.2     6  168.   123  3.92  3.44  18.3     1     0     4     4
## # ... with 22 more rows
\end{verbatim}

\begin{Shaded}
\begin{Highlighting}[]
\NormalTok{carros }\OperatorTok\StringTok{ }
\StringTok{  }\KeywordTok{add_row}\NormalTok{(}\DataTypeTok{.before =} \DecValTok{1}\NormalTok{)}
\end{Highlighting}
\end{Shaded}

\begin{verbatim}
## # A tibble: 33 x 11
##      mpg   cyl  disp    hp  drat    wt  qsec    vs    am  gear  carb
##  * <dbl> <dbl> <dbl> <dbl> <dbl> <dbl> <dbl> <dbl> <dbl> <dbl> <dbl>
##  1  NA      NA   NA     NA NA    NA     NA      NA    NA    NA    NA
##  2  21       6  160    110  3.9   2.62  16.5     0     1     4     4
##  3  21       6  160    110  3.9   2.88  17.0     0     1     4     4
##  4  22.8     4  108     93  3.85  2.32  18.6     1     1     4     1
##  5  21.4     6  258    110  3.08  3.22  19.4     1     0     3     1
##  6  18.7     8  360    175  3.15  3.44  17.0     0     0     3     2
##  7  18.1     6  225    105  2.76  3.46  20.2     1     0     3     1
##  8  14.3     8  360    245  3.21  3.57  15.8     0     0     3     4
##  9  24.4     4  147.    62  3.69  3.19  20       1     0     4     2
## 10  22.8     4  141.    95  3.92  3.15  22.9     1     0     4     2
## # ... with 23 more rows
\end{verbatim}

\hypertarget{remove_rownames-e-column_to_rownames---remove-os-nomes-das-colunas-e-adiciona-um-nome-as-colunas}{%
\section{remove\_rownames() e column\_to\_rownames() - Remove os nomes das colunas e adiciona um nome as colunas}\label{remove_rownames-e-column_to_rownames---remove-os-nomes-das-colunas-e-adiciona-um-nome-as-colunas}}

\begin{Shaded}
\begin{Highlighting}[]
\NormalTok{carros }\OperatorTok
\StringTok{  }\KeywordTok{add_column}\NormalTok{(}\DataTypeTok{teste =} \KeywordTok{c}\NormalTok{(}\StringTok{"a"}\NormalTok{,}\StringTok{"b"}\NormalTok{,}\StringTok{"c"}\NormalTok{,}\StringTok{"d"}\NormalTok{,}\StringTok{"e"}\NormalTok{,}\StringTok{"f"}\NormalTok{,}\StringTok{"g"}\NormalTok{,}\StringTok{"h"}\NormalTok{,}\StringTok{"i"}\NormalTok{,}\StringTok{"j"}\NormalTok{,}\StringTok{"k"}\NormalTok{,}\StringTok{"l"}\NormalTok{,}\StringTok{"m"}\NormalTok{,}\StringTok{"n"}\NormalTok{,}\StringTok{"o"}\NormalTok{,}\StringTok{"p"}\NormalTok{,}\StringTok{"q"}\NormalTok{,}\StringTok{"r"}\NormalTok{,}\StringTok{"s"}\NormalTok{,}\StringTok{"t"}\NormalTok{,}\StringTok{"u"}\NormalTok{,}\StringTok{"v"}\NormalTok{,}\StringTok{"w"}\NormalTok{,}\StringTok{"x"}\NormalTok{,}\StringTok{"y"}\NormalTok{,}\StringTok{"z"}\NormalTok{,}\StringTok{"aa"}\NormalTok{,}\StringTok{"bb"}\NormalTok{,}\StringTok{"cc"}\NormalTok{,}\StringTok{"dd"}\NormalTok{,}\StringTok{"ee"}\NormalTok{,}\StringTok{"ff"}\NormalTok{)) }\OperatorTok\StringTok{ }
\StringTok{  }\KeywordTok{remove_rownames}\NormalTok{() }\OperatorTok
\StringTok{  }\KeywordTok{column_to_rownames}\NormalTok{(}\StringTok{"teste"}\NormalTok{)}
\end{Highlighting}
\end{Shaded}

\begin{verbatim}
##     mpg cyl  disp  hp drat    wt  qsec vs am gear carb
## a  21.0   6 160.0 110 3.90 2.620 16.46  0  1    4    4
## b  21.0   6 160.0 110 3.90 2.875 17.02  0  1    4    4
## c  22.8   4 108.0  93 3.85 2.320 18.61  1  1    4    1
## d  21.4   6 258.0 110 3.08 3.215 19.44  1  0    3    1
## e  18.7   8 360.0 175 3.15 3.440 17.02  0  0    3    2
## f  18.1   6 225.0 105 2.76 3.460 20.22  1  0    3    1
## g  14.3   8 360.0 245 3.21 3.570 15.84  0  0    3    4
## h  24.4   4 146.7  62 3.69 3.190 20.00  1  0    4    2
## i  22.8   4 140.8  95 3.92 3.150 22.90  1  0    4    2
## j  19.2   6 167.6 123 3.92 3.440 18.30  1  0    4    4
## k  17.8   6 167.6 123 3.92 3.440 18.90  1  0    4    4
## l  16.4   8 275.8 180 3.07 4.070 17.40  0  0    3    3
## m  17.3   8 275.8 180 3.07 3.730 17.60  0  0    3    3
## n  15.2   8 275.8 180 3.07 3.780 18.00  0  0    3    3
## o  10.4   8 472.0 205 2.93 5.250 17.98  0  0    3    4
## p  10.4   8 460.0 215 3.00 5.424 17.82  0  0    3    4
## q  14.7   8 440.0 230 3.23 5.345 17.42  0  0    3    4
## r  32.4   4  78.7  66 4.08 2.200 19.47  1  1    4    1
## s  30.4   4  75.7  52 4.93 1.615 18.52  1  1    4    2
## t  33.9   4  71.1  65 4.22 1.835 19.90  1  1    4    1
## u  21.5   4 120.1  97 3.70 2.465 20.01  1  0    3    1
## v  15.5   8 318.0 150 2.76 3.520 16.87  0  0    3    2
## w  15.2   8 304.0 150 3.15 3.435 17.30  0  0    3    2
## x  13.3   8 350.0 245 3.73 3.840 15.41  0  0    3    4
## y  19.2   8 400.0 175 3.08 3.845 17.05  0  0    3    2
## z  27.3   4  79.0  66 4.08 1.935 18.90  1  1    4    1
## aa 26.0   4 120.3  91 4.43 2.140 16.70  0  1    5    2
## bb 30.4   4  95.1 113 3.77 1.513 16.90  1  1    5    2
## cc 15.8   8 351.0 264 4.22 3.170 14.50  0  1    5    4
## dd 19.7   6 145.0 175 3.62 2.770 15.50  0  1    5    6
## ee 15.0   8 301.0 335 3.54 3.570 14.60  0  1    5    8
## ff 21.4   4 121.0 109 4.11 2.780 18.60  1  1    4    2
\end{verbatim}

\hypertarget{rownames_to_column---adiciona-o-nome-das-linhas-como-uma-coluna}{%
\section{rownames\_to\_column() - adiciona o nome das linhas como uma coluna}\label{rownames_to_column---adiciona-o-nome-das-linhas-como-uma-coluna}}

\begin{Shaded}
\begin{Highlighting}[]
\NormalTok{carros }\OperatorTok\StringTok{ }
\StringTok{  }\KeywordTok{rownames_to_column}\NormalTok{(}\DataTypeTok{var =} \StringTok{"teste"}\NormalTok{)}
\end{Highlighting}
\end{Shaded}

\begin{verbatim}
## # A tibble: 32 x 12
##    teste         mpg   cyl  disp    hp  drat    wt  qsec    vs    am  gear  carb
##    <chr>       <dbl> <dbl> <dbl> <dbl> <dbl> <dbl> <dbl> <dbl> <dbl> <dbl> <dbl>
##  1 Mazda RX4    21       6  160    110  3.9   2.62  16.5     0     1     4     4
##  2 Mazda RX4 ~  21       6  160    110  3.9   2.88  17.0     0     1     4     4
##  3 Datsun 710   22.8     4  108     93  3.85  2.32  18.6     1     1     4     1
##  4 Hornet 4 D~  21.4     6  258    110  3.08  3.22  19.4     1     0     3     1
##  5 Hornet Spo~  18.7     8  360    175  3.15  3.44  17.0     0     0     3     2
##  6 Valiant      18.1     6  225    105  2.76  3.46  20.2     1     0     3     1
##  7 Duster 360   14.3     8  360    245  3.21  3.57  15.8     0     0     3     4
##  8 Merc 240D    24.4     4  147.    62  3.69  3.19  20       1     0     4     2
##  9 Merc 230     22.8     4  141.    95  3.92  3.15  22.9     1     0     4     2
## 10 Merc 280     19.2     6  168.   123  3.92  3.44  18.3     1     0     4     4
## # ... with 22 more rows
\end{verbatim}

\hypertarget{subsetting---forma-de-escrita-que-permite-cortarselecionar-a-planilha-em-funuxe7uxe3o-das-linhas-e-colunas}{%
\section{subsetting - forma de escrita que permite cortar/selecionar a planilha em função das linhas e colunas}\label{subsetting---forma-de-escrita-que-permite-cortarselecionar-a-planilha-em-funuxe7uxe3o-das-linhas-e-colunas}}

\begin{Shaded}
\begin{Highlighting}[]
\NormalTok{carros[}\DecValTok{1}\NormalTok{, ]}
\end{Highlighting}
\end{Shaded}

\begin{verbatim}
## # A tibble: 1 x 11
##     mpg   cyl  disp    hp  drat    wt  qsec    vs    am  gear  carb
##   <dbl> <dbl> <dbl> <dbl> <dbl> <dbl> <dbl> <dbl> <dbl> <dbl> <dbl>
## 1    21     6   160   110   3.9  2.62  16.5     0     1     4     4
\end{verbatim}

\begin{Shaded}
\begin{Highlighting}[]
\NormalTok{carros[}\DecValTok{2}\NormalTok{, ]}
\end{Highlighting}
\end{Shaded}

\begin{verbatim}
## # A tibble: 1 x 11
##     mpg   cyl  disp    hp  drat    wt  qsec    vs    am  gear  carb
##   <dbl> <dbl> <dbl> <dbl> <dbl> <dbl> <dbl> <dbl> <dbl> <dbl> <dbl>
## 1    21     6   160   110   3.9  2.88  17.0     0     1     4     4
\end{verbatim}

\begin{Shaded}
\begin{Highlighting}[]
\NormalTok{carros[}\DecValTok{1}\OperatorTok{:}\DecValTok{2}\NormalTok{, ]}
\end{Highlighting}
\end{Shaded}

\begin{verbatim}
## # A tibble: 2 x 11
##     mpg   cyl  disp    hp  drat    wt  qsec    vs    am  gear  carb
##   <dbl> <dbl> <dbl> <dbl> <dbl> <dbl> <dbl> <dbl> <dbl> <dbl> <dbl>
## 1    21     6   160   110   3.9  2.62  16.5     0     1     4     4
## 2    21     6   160   110   3.9  2.88  17.0     0     1     4     4
\end{verbatim}

\begin{Shaded}
\begin{Highlighting}[]
\NormalTok{carros[, }\DecValTok{1}\NormalTok{]}
\end{Highlighting}
\end{Shaded}

\begin{verbatim}
## # A tibble: 32 x 1
##      mpg
##    <dbl>
##  1  21  
##  2  21  
##  3  22.8
##  4  21.4
##  5  18.7
##  6  18.1
##  7  14.3
##  8  24.4
##  9  22.8
## 10  19.2
## # ... with 22 more rows
\end{verbatim}

\begin{Shaded}
\begin{Highlighting}[]
\NormalTok{carros[, }\StringTok{"mpg"}\NormalTok{]}
\end{Highlighting}
\end{Shaded}

\begin{verbatim}
## # A tibble: 32 x 1
##      mpg
##    <dbl>
##  1  21  
##  2  21  
##  3  22.8
##  4  21.4
##  5  18.7
##  6  18.1
##  7  14.3
##  8  24.4
##  9  22.8
## 10  19.2
## # ... with 22 more rows
\end{verbatim}

\begin{Shaded}
\begin{Highlighting}[]
\NormalTok{carros[[}\DecValTok{1}\NormalTok{]]}
\end{Highlighting}
\end{Shaded}

\begin{verbatim}
##  [1] 21.0 21.0 22.8 21.4 18.7 18.1 14.3 24.4 22.8 19.2 17.8 16.4 17.3 15.2 10.4
## [16] 10.4 14.7 32.4 30.4 33.9 21.5 15.5 15.2 13.3 19.2 27.3 26.0 30.4 15.8 19.7
## [31] 15.0 21.4
\end{verbatim}

\begin{Shaded}
\begin{Highlighting}[]
\NormalTok{carros[[}\StringTok{"mpg"}\NormalTok{]]}
\end{Highlighting}
\end{Shaded}

\begin{verbatim}
##  [1] 21.0 21.0 22.8 21.4 18.7 18.1 14.3 24.4 22.8 19.2 17.8 16.4 17.3 15.2 10.4
## [16] 10.4 14.7 32.4 30.4 33.9 21.5 15.5 15.2 13.3 19.2 27.3 26.0 30.4 15.8 19.7
## [31] 15.0 21.4
\end{verbatim}

\begin{Shaded}
\begin{Highlighting}[]
\NormalTok{carros[[}\StringTok{"mpg"}\NormalTok{]]}
\end{Highlighting}
\end{Shaded}

\begin{verbatim}
##  [1] 21.0 21.0 22.8 21.4 18.7 18.1 14.3 24.4 22.8 19.2 17.8 16.4 17.3 15.2 10.4
## [16] 10.4 14.7 32.4 30.4 33.9 21.5 15.5 15.2 13.3 19.2 27.3 26.0 30.4 15.8 19.7
## [31] 15.0 21.4
\end{verbatim}

\begin{Shaded}
\begin{Highlighting}[]
\NormalTok{carros}\OperatorTok{$}\NormalTok{mpg}
\end{Highlighting}
\end{Shaded}

\begin{verbatim}
##  [1] 21.0 21.0 22.8 21.4 18.7 18.1 14.3 24.4 22.8 19.2 17.8 16.4 17.3 15.2 10.4
## [16] 10.4 14.7 32.4 30.4 33.9 21.5 15.5 15.2 13.3 19.2 27.3 26.0 30.4 15.8 19.7
## [31] 15.0 21.4
\end{verbatim}

\begin{Shaded}
\begin{Highlighting}[]
\NormalTok{carros[, }\KeywordTok{c}\NormalTok{(}\StringTok{"mpg"}\NormalTok{, }\StringTok{"cyl"}\NormalTok{)]}
\end{Highlighting}
\end{Shaded}

\begin{verbatim}
## # A tibble: 32 x 2
##      mpg   cyl
##    <dbl> <dbl>
##  1  21       6
##  2  21       6
##  3  22.8     4
##  4  21.4     6
##  5  18.7     8
##  6  18.1     6
##  7  14.3     8
##  8  24.4     4
##  9  22.8     4
## 10  19.2     6
## # ... with 22 more rows
\end{verbatim}

\hypertarget{dplyr}{%
\chapter{\texorpdfstring{\emph{dplyr}}{dplyr}}\label{dplyr}}

\hypertarget{count-e-add_count---conta-o-nuxfamero-de-linhas-em-funuxe7uxe3o-da-variuxe1vel-especificada-adiciona-uma-nova-coluna-com-a-contagem-do-nuxfamero-de-linhas-em-funuxe7uxe3o-da-variuxe1vel-especificada}{%
\section{count() e add\_count() - conta o número de linhas em função da variável especificada; adiciona uma nova coluna com a contagem do número de linhas em função da variável especificada}\label{count-e-add_count---conta-o-nuxfamero-de-linhas-em-funuxe7uxe3o-da-variuxe1vel-especificada-adiciona-uma-nova-coluna-com-a-contagem-do-nuxfamero-de-linhas-em-funuxe7uxe3o-da-variuxe1vel-especificada}}

\begin{Shaded}
\begin{Highlighting}[]
\NormalTok{flores }\OperatorTok\StringTok{ }
\StringTok{  }\KeywordTok{count}\NormalTok{(Species)}
\end{Highlighting}
\end{Shaded}

\begin{verbatim}
## # A tibble: 3 x 2
##   Species        n
##   <fct>      <int>
## 1 setosa        50
## 2 versicolor    50
## 3 virginica     50
\end{verbatim}

\begin{Shaded}
\begin{Highlighting}[]
\NormalTok{flores }\OperatorTok\StringTok{ }
\StringTok{  }\KeywordTok{add_count}\NormalTok{(Species)}
\end{Highlighting}
\end{Shaded}

\begin{verbatim}
## # A tibble: 150 x 6
##    Sepal.Length Sepal.Width Petal.Length Petal.Width Species     n
##           <dbl>       <dbl>        <dbl>       <dbl> <fct>   <int>
##  1          5.1         3.5          1.4         0.2 setosa     50
##  2          4.9         3            1.4         0.2 setosa     50
##  3          4.7         3.2          1.3         0.2 setosa     50
##  4          4.6         3.1          1.5         0.2 setosa     50
##  5          5           3.6          1.4         0.2 setosa     50
##  6          5.4         3.9          1.7         0.4 setosa     50
##  7          4.6         3.4          1.4         0.3 setosa     50
##  8          5           3.4          1.5         0.2 setosa     50
##  9          4.4         2.9          1.4         0.2 setosa     50
## 10          4.9         3.1          1.5         0.1 setosa     50
## # ... with 140 more rows
\end{verbatim}

\hypertarget{arrange---rearranja-as-linhas-em-funuxe7uxe3o-da-variuxe1vel-especificada}{%
\section{arrange() - rearranja as linhas em função da variável especificada}\label{arrange---rearranja-as-linhas-em-funuxe7uxe3o-da-variuxe1vel-especificada}}

\begin{Shaded}
\begin{Highlighting}[]
\NormalTok{flores }\OperatorTok\StringTok{ }
\StringTok{  }\KeywordTok{arrange}\NormalTok{(Species)}
\end{Highlighting}
\end{Shaded}

\begin{verbatim}
## # A tibble: 150 x 5
##    Sepal.Length Sepal.Width Petal.Length Petal.Width Species
##           <dbl>       <dbl>        <dbl>       <dbl> <fct>  
##  1          5.1         3.5          1.4         0.2 setosa 
##  2          4.9         3            1.4         0.2 setosa 
##  3          4.7         3.2          1.3         0.2 setosa 
##  4          4.6         3.1          1.5         0.2 setosa 
##  5          5           3.6          1.4         0.2 setosa 
##  6          5.4         3.9          1.7         0.4 setosa 
##  7          4.6         3.4          1.4         0.3 setosa 
##  8          5           3.4          1.5         0.2 setosa 
##  9          4.4         2.9          1.4         0.2 setosa 
## 10          4.9         3.1          1.5         0.1 setosa 
## # ... with 140 more rows
\end{verbatim}

\begin{Shaded}
\begin{Highlighting}[]
\NormalTok{flores }\OperatorTok\StringTok{ }
\StringTok{  }\KeywordTok{arrange}\NormalTok{(}\KeywordTok{desc}\NormalTok{(Species))}
\end{Highlighting}
\end{Shaded}

\begin{verbatim}
## # A tibble: 150 x 5
##    Sepal.Length Sepal.Width Petal.Length Petal.Width Species  
##           <dbl>       <dbl>        <dbl>       <dbl> <fct>    
##  1          6.3         3.3          6           2.5 virginica
##  2          5.8         2.7          5.1         1.9 virginica
##  3          7.1         3            5.9         2.1 virginica
##  4          6.3         2.9          5.6         1.8 virginica
##  5          6.5         3            5.8         2.2 virginica
##  6          7.6         3            6.6         2.1 virginica
##  7          4.9         2.5          4.5         1.7 virginica
##  8          7.3         2.9          6.3         1.8 virginica
##  9          6.7         2.5          5.8         1.8 virginica
## 10          7.2         3.6          6.1         2.5 virginica
## # ... with 140 more rows
\end{verbatim}

\begin{Shaded}
\begin{Highlighting}[]
\NormalTok{flores }\OperatorTok\StringTok{ }
\StringTok{  }\KeywordTok{arrange}\NormalTok{(}\KeywordTok{factor}\NormalTok{(Species, }\DataTypeTok{levels =} \KeywordTok{c}\NormalTok{(}\StringTok{"versicolor"}\NormalTok{, }\StringTok{"setosa"}\NormalTok{, }\StringTok{"virginica"}\NormalTok{)))}
\end{Highlighting}
\end{Shaded}

\begin{verbatim}
## # A tibble: 150 x 5
##    Sepal.Length Sepal.Width Petal.Length Petal.Width Species   
##           <dbl>       <dbl>        <dbl>       <dbl> <fct>     
##  1          7           3.2          4.7         1.4 versicolor
##  2          6.4         3.2          4.5         1.5 versicolor
##  3          6.9         3.1          4.9         1.5 versicolor
##  4          5.5         2.3          4           1.3 versicolor
##  5          6.5         2.8          4.6         1.5 versicolor
##  6          5.7         2.8          4.5         1.3 versicolor
##  7          6.3         3.3          4.7         1.6 versicolor
##  8          4.9         2.4          3.3         1   versicolor
##  9          6.6         2.9          4.6         1.3 versicolor
## 10          5.2         2.7          3.9         1.4 versicolor
## # ... with 140 more rows
\end{verbatim}

\hypertarget{bind_cols---une-planilhas-em-colunas}{%
\section{bind\_cols() - une planilhas em colunas}\label{bind_cols---une-planilhas-em-colunas}}

\begin{Shaded}
\begin{Highlighting}[]
\NormalTok{parte1 <-}\StringTok{ }\NormalTok{flores[,}\DecValTok{1}\OperatorTok{:}\DecValTok{2}\NormalTok{]}
\NormalTok{parte2 <-}\StringTok{ }\NormalTok{flores[,}\DecValTok{3}\OperatorTok{:}\DecValTok{5}\NormalTok{]}

\KeywordTok{bind_cols}\NormalTok{(parte1, parte2)}
\end{Highlighting}
\end{Shaded}

\begin{verbatim}
## # A tibble: 150 x 5
##    Sepal.Length Sepal.Width Petal.Length Petal.Width Species
##           <dbl>       <dbl>        <dbl>       <dbl> <fct>  
##  1          5.1         3.5          1.4         0.2 setosa 
##  2          4.9         3            1.4         0.2 setosa 
##  3          4.7         3.2          1.3         0.2 setosa 
##  4          4.6         3.1          1.5         0.2 setosa 
##  5          5           3.6          1.4         0.2 setosa 
##  6          5.4         3.9          1.7         0.4 setosa 
##  7          4.6         3.4          1.4         0.3 setosa 
##  8          5           3.4          1.5         0.2 setosa 
##  9          4.4         2.9          1.4         0.2 setosa 
## 10          4.9         3.1          1.5         0.1 setosa 
## # ... with 140 more rows
\end{verbatim}

\hypertarget{bind_rows---une-planilhas-em-linhas}{%
\section{bind\_rows() - une planilhas em linhas}\label{bind_rows---une-planilhas-em-linhas}}

\begin{Shaded}
\begin{Highlighting}[]
\NormalTok{parte1 <-}\StringTok{ }\NormalTok{flores[}\DecValTok{1}\OperatorTok{:}\DecValTok{10}\NormalTok{,]}
\NormalTok{parte2 <-}\StringTok{ }\NormalTok{flores[}\DecValTok{11}\OperatorTok{:}\DecValTok{150}\NormalTok{,]}

\KeywordTok{bind_rows}\NormalTok{(parte1, parte2)}
\end{Highlighting}
\end{Shaded}

\begin{verbatim}
## # A tibble: 150 x 5
##    Sepal.Length Sepal.Width Petal.Length Petal.Width Species
##           <dbl>       <dbl>        <dbl>       <dbl> <fct>  
##  1          5.1         3.5          1.4         0.2 setosa 
##  2          4.9         3            1.4         0.2 setosa 
##  3          4.7         3.2          1.3         0.2 setosa 
##  4          4.6         3.1          1.5         0.2 setosa 
##  5          5           3.6          1.4         0.2 setosa 
##  6          5.4         3.9          1.7         0.4 setosa 
##  7          4.6         3.4          1.4         0.3 setosa 
##  8          5           3.4          1.5         0.2 setosa 
##  9          4.4         2.9          1.4         0.2 setosa 
## 10          4.9         3.1          1.5         0.1 setosa 
## # ... with 140 more rows
\end{verbatim}

\hypertarget{distinct---remove-as-linhas-que-suxe3o-exatamente-iguais}{%
\section{distinct() - remove as linhas que são exatamente iguais}\label{distinct---remove-as-linhas-que-suxe3o-exatamente-iguais}}

\begin{Shaded}
\begin{Highlighting}[]
\NormalTok{flores}
\end{Highlighting}
\end{Shaded}

\begin{verbatim}
## # A tibble: 150 x 5
##    Sepal.Length Sepal.Width Petal.Length Petal.Width Species
##           <dbl>       <dbl>        <dbl>       <dbl> <fct>  
##  1          5.1         3.5          1.4         0.2 setosa 
##  2          4.9         3            1.4         0.2 setosa 
##  3          4.7         3.2          1.3         0.2 setosa 
##  4          4.6         3.1          1.5         0.2 setosa 
##  5          5           3.6          1.4         0.2 setosa 
##  6          5.4         3.9          1.7         0.4 setosa 
##  7          4.6         3.4          1.4         0.3 setosa 
##  8          5           3.4          1.5         0.2 setosa 
##  9          4.4         2.9          1.4         0.2 setosa 
## 10          4.9         3.1          1.5         0.1 setosa 
## # ... with 140 more rows
\end{verbatim}

\begin{Shaded}
\begin{Highlighting}[]
\NormalTok{flores }\OperatorTok\StringTok{ }
\StringTok{  }\KeywordTok{distinct}\NormalTok{()}
\end{Highlighting}
\end{Shaded}

\begin{verbatim}
## # A tibble: 149 x 5
##    Sepal.Length Sepal.Width Petal.Length Petal.Width Species
##           <dbl>       <dbl>        <dbl>       <dbl> <fct>  
##  1          5.1         3.5          1.4         0.2 setosa 
##  2          4.9         3            1.4         0.2 setosa 
##  3          4.7         3.2          1.3         0.2 setosa 
##  4          4.6         3.1          1.5         0.2 setosa 
##  5          5           3.6          1.4         0.2 setosa 
##  6          5.4         3.9          1.7         0.4 setosa 
##  7          4.6         3.4          1.4         0.3 setosa 
##  8          5           3.4          1.5         0.2 setosa 
##  9          4.4         2.9          1.4         0.2 setosa 
## 10          4.9         3.1          1.5         0.1 setosa 
## # ... with 139 more rows
\end{verbatim}

\begin{Shaded}
\begin{Highlighting}[]
\NormalTok{flores }\OperatorTok\StringTok{ }
\StringTok{  }\KeywordTok{distinct}\NormalTok{() }\OperatorTok\StringTok{ }
\StringTok{  }\KeywordTok{count}\NormalTok{(Species)}
\end{Highlighting}
\end{Shaded}

\begin{verbatim}
## # A tibble: 3 x 2
##   Species        n
##   <fct>      <int>
## 1 setosa        50
## 2 versicolor    50
## 3 virginica     49
\end{verbatim}

\hypertarget{filter---filtra-as-linhas-que-satisfauxe7uxe3o-alguma-condiuxe7uxe3o}{%
\section{filter() - filtra as linhas que satisfação alguma condição}\label{filter---filtra-as-linhas-que-satisfauxe7uxe3o-alguma-condiuxe7uxe3o}}

\begin{Shaded}
\begin{Highlighting}[]
\NormalTok{flores }\OperatorTok\StringTok{ }
\StringTok{  }\KeywordTok{filter}\NormalTok{(Species }\OperatorTok{==}\StringTok{ "setosa"}\NormalTok{)}
\end{Highlighting}
\end{Shaded}

\begin{verbatim}
## # A tibble: 50 x 5
##    Sepal.Length Sepal.Width Petal.Length Petal.Width Species
##           <dbl>       <dbl>        <dbl>       <dbl> <fct>  
##  1          5.1         3.5          1.4         0.2 setosa 
##  2          4.9         3            1.4         0.2 setosa 
##  3          4.7         3.2          1.3         0.2 setosa 
##  4          4.6         3.1          1.5         0.2 setosa 
##  5          5           3.6          1.4         0.2 setosa 
##  6          5.4         3.9          1.7         0.4 setosa 
##  7          4.6         3.4          1.4         0.3 setosa 
##  8          5           3.4          1.5         0.2 setosa 
##  9          4.4         2.9          1.4         0.2 setosa 
## 10          4.9         3.1          1.5         0.1 setosa 
## # ... with 40 more rows
\end{verbatim}

\begin{Shaded}
\begin{Highlighting}[]
\NormalTok{flores }\OperatorTok\StringTok{ }
\StringTok{  }\KeywordTok{filter}\NormalTok{(Species }\OperatorTok\StringTok{ }\KeywordTok{c}\NormalTok{(}\StringTok{"setosa"}\NormalTok{, }\StringTok{"virginica"}\NormalTok{))}
\end{Highlighting}
\end{Shaded}

\begin{verbatim}
## # A tibble: 100 x 5
##    Sepal.Length Sepal.Width Petal.Length Petal.Width Species
##           <dbl>       <dbl>        <dbl>       <dbl> <fct>  
##  1          5.1         3.5          1.4         0.2 setosa 
##  2          4.9         3            1.4         0.2 setosa 
##  3          4.7         3.2          1.3         0.2 setosa 
##  4          4.6         3.1          1.5         0.2 setosa 
##  5          5           3.6          1.4         0.2 setosa 
##  6          5.4         3.9          1.7         0.4 setosa 
##  7          4.6         3.4          1.4         0.3 setosa 
##  8          5           3.4          1.5         0.2 setosa 
##  9          4.4         2.9          1.4         0.2 setosa 
## 10          4.9         3.1          1.5         0.1 setosa 
## # ... with 90 more rows
\end{verbatim}

\begin{Shaded}
\begin{Highlighting}[]
\NormalTok{flores }\OperatorTok\StringTok{ }
\StringTok{  }\KeywordTok{filter}\NormalTok{(Sepal.Length }\OperatorTok{>=}\StringTok{ }\KeywordTok{mean}\NormalTok{(Sepal.Length))}
\end{Highlighting}
\end{Shaded}

\begin{verbatim}
## # A tibble: 70 x 5
##    Sepal.Length Sepal.Width Petal.Length Petal.Width Species   
##           <dbl>       <dbl>        <dbl>       <dbl> <fct>     
##  1          7           3.2          4.7         1.4 versicolor
##  2          6.4         3.2          4.5         1.5 versicolor
##  3          6.9         3.1          4.9         1.5 versicolor
##  4          6.5         2.8          4.6         1.5 versicolor
##  5          6.3         3.3          4.7         1.6 versicolor
##  6          6.6         2.9          4.6         1.3 versicolor
##  7          5.9         3            4.2         1.5 versicolor
##  8          6           2.2          4           1   versicolor
##  9          6.1         2.9          4.7         1.4 versicolor
## 10          6.7         3.1          4.4         1.4 versicolor
## # ... with 60 more rows
\end{verbatim}

\begin{Shaded}
\begin{Highlighting}[]
\NormalTok{flores }\OperatorTok\StringTok{ }
\StringTok{  }\KeywordTok{filter}\NormalTok{(Species }\OperatorTok{==}\StringTok{ "setosa"}\NormalTok{) }\OperatorTok\StringTok{ }
\StringTok{  }\KeywordTok{filter}\NormalTok{(Sepal.Length }\OperatorTok{>=}\StringTok{ }\KeywordTok{mean}\NormalTok{(Sepal.Length))}
\end{Highlighting}
\end{Shaded}

\begin{verbatim}
## # A tibble: 22 x 5
##    Sepal.Length Sepal.Width Petal.Length Petal.Width Species
##           <dbl>       <dbl>        <dbl>       <dbl> <fct>  
##  1          5.1         3.5          1.4         0.2 setosa 
##  2          5.4         3.9          1.7         0.4 setosa 
##  3          5.4         3.7          1.5         0.2 setosa 
##  4          5.8         4            1.2         0.2 setosa 
##  5          5.7         4.4          1.5         0.4 setosa 
##  6          5.4         3.9          1.3         0.4 setosa 
##  7          5.1         3.5          1.4         0.3 setosa 
##  8          5.7         3.8          1.7         0.3 setosa 
##  9          5.1         3.8          1.5         0.3 setosa 
## 10          5.4         3.4          1.7         0.2 setosa 
## # ... with 12 more rows
\end{verbatim}

\begin{Shaded}
\begin{Highlighting}[]
\NormalTok{flores }\OperatorTok\StringTok{ }
\StringTok{  }\KeywordTok{filter}\NormalTok{(Petal.Length }\OperatorTok{>=}\StringTok{ }\KeywordTok{mean}\NormalTok{(Petal.Length) }\OperatorTok{&}\StringTok{ }\NormalTok{Petal.Width }\OperatorTok{>=}\StringTok{ }\KeywordTok{mean}\NormalTok{(Petal.Width))}
\end{Highlighting}
\end{Shaded}

\begin{verbatim}
## # A tibble: 89 x 5
##    Sepal.Length Sepal.Width Petal.Length Petal.Width Species   
##           <dbl>       <dbl>        <dbl>       <dbl> <fct>     
##  1          7           3.2          4.7         1.4 versicolor
##  2          6.4         3.2          4.5         1.5 versicolor
##  3          6.9         3.1          4.9         1.5 versicolor
##  4          5.5         2.3          4           1.3 versicolor
##  5          6.5         2.8          4.6         1.5 versicolor
##  6          5.7         2.8          4.5         1.3 versicolor
##  7          6.3         3.3          4.7         1.6 versicolor
##  8          6.6         2.9          4.6         1.3 versicolor
##  9          5.2         2.7          3.9         1.4 versicolor
## 10          5.9         3            4.2         1.5 versicolor
## # ... with 79 more rows
\end{verbatim}

\begin{Shaded}
\begin{Highlighting}[]
\NormalTok{flores }\OperatorTok\StringTok{ }
\StringTok{  }\KeywordTok{filter}\NormalTok{(Petal.Length }\OperatorTok{>=}\StringTok{ }\KeywordTok{mean}\NormalTok{(Petal.Length) }\OperatorTok{&}\StringTok{ }\NormalTok{Petal.Width }\OperatorTok{>=}\StringTok{ }\KeywordTok{mean}\NormalTok{(Petal.Width)) }\OperatorTok\StringTok{ }
\StringTok{  }\KeywordTok{count}\NormalTok{(Species)}
\end{Highlighting}
\end{Shaded}

\begin{verbatim}
## # A tibble: 2 x 2
##   Species        n
##   <fct>      <int>
## 1 versicolor    39
## 2 virginica     50
\end{verbatim}

\hypertarget{group_by---agrupa-as-linhas-em-funuxe7uxe3o-dos-valores-de-alguma-variuxe1vel}{%
\section{group\_by() - agrupa as linhas em função dos valores de alguma variável}\label{group_by---agrupa-as-linhas-em-funuxe7uxe3o-dos-valores-de-alguma-variuxe1vel}}

\begin{Shaded}
\begin{Highlighting}[]
\CommentTok{#Ver esta função juntamente com mutate(), transmute() e summarise()}

\NormalTok{flores }\OperatorTok\StringTok{ }
\StringTok{  }\KeywordTok{group_by}\NormalTok{(Species)}
\end{Highlighting}
\end{Shaded}

\begin{verbatim}
## # A tibble: 150 x 5
## # Groups:   Species [3]
##    Sepal.Length Sepal.Width Petal.Length Petal.Width Species
##           <dbl>       <dbl>        <dbl>       <dbl> <fct>  
##  1          5.1         3.5          1.4         0.2 setosa 
##  2          4.9         3            1.4         0.2 setosa 
##  3          4.7         3.2          1.3         0.2 setosa 
##  4          4.6         3.1          1.5         0.2 setosa 
##  5          5           3.6          1.4         0.2 setosa 
##  6          5.4         3.9          1.7         0.4 setosa 
##  7          4.6         3.4          1.4         0.3 setosa 
##  8          5           3.4          1.5         0.2 setosa 
##  9          4.4         2.9          1.4         0.2 setosa 
## 10          4.9         3.1          1.5         0.1 setosa 
## # ... with 140 more rows
\end{verbatim}

\hypertarget{mutate-e-transmute---adiciona-novas-variuxe1veis-e-preserva-as-existentes-adiciona-novas-variuxe1veis-e-nuxe3o-preserva-as-existentes}{%
\section{mutate() e transmute() - adiciona novas variáveis e preserva as existentes; adiciona novas variáveis e não preserva as existentes}\label{mutate-e-transmute---adiciona-novas-variuxe1veis-e-preserva-as-existentes-adiciona-novas-variuxe1veis-e-nuxe3o-preserva-as-existentes}}

\begin{Shaded}
\begin{Highlighting}[]
\NormalTok{flores }\OperatorTok
\StringTok{  }\KeywordTok{mutate}\NormalTok{(}\DataTypeTok{teste =}\NormalTok{ Petal.Length }\OperatorTok{+}\StringTok{ }\NormalTok{Sepal.Length)}
\end{Highlighting}
\end{Shaded}

\begin{verbatim}
## # A tibble: 150 x 6
##    Sepal.Length Sepal.Width Petal.Length Petal.Width Species teste
##           <dbl>       <dbl>        <dbl>       <dbl> <fct>   <dbl>
##  1          5.1         3.5          1.4         0.2 setosa    6.5
##  2          4.9         3            1.4         0.2 setosa    6.3
##  3          4.7         3.2          1.3         0.2 setosa    6  
##  4          4.6         3.1          1.5         0.2 setosa    6.1
##  5          5           3.6          1.4         0.2 setosa    6.4
##  6          5.4         3.9          1.7         0.4 setosa    7.1
##  7          4.6         3.4          1.4         0.3 setosa    6  
##  8          5           3.4          1.5         0.2 setosa    6.5
##  9          4.4         2.9          1.4         0.2 setosa    5.8
## 10          4.9         3.1          1.5         0.1 setosa    6.4
## # ... with 140 more rows
\end{verbatim}

\begin{Shaded}
\begin{Highlighting}[]
\NormalTok{flores }\OperatorTok
\StringTok{  }\KeywordTok{group_by}\NormalTok{(Species) }\OperatorTok\StringTok{ }
\StringTok{  }\KeywordTok{mutate}\NormalTok{(}\DataTypeTok{PL_media =} \KeywordTok{mean}\NormalTok{(Petal.Length))}
\end{Highlighting}
\end{Shaded}

\begin{verbatim}
## # A tibble: 150 x 6
## # Groups:   Species [3]
##    Sepal.Length Sepal.Width Petal.Length Petal.Width Species PL_media
##           <dbl>       <dbl>        <dbl>       <dbl> <fct>      <dbl>
##  1          5.1         3.5          1.4         0.2 setosa      1.46
##  2          4.9         3            1.4         0.2 setosa      1.46
##  3          4.7         3.2          1.3         0.2 setosa      1.46
##  4          4.6         3.1          1.5         0.2 setosa      1.46
##  5          5           3.6          1.4         0.2 setosa      1.46
##  6          5.4         3.9          1.7         0.4 setosa      1.46
##  7          4.6         3.4          1.4         0.3 setosa      1.46
##  8          5           3.4          1.5         0.2 setosa      1.46
##  9          4.4         2.9          1.4         0.2 setosa      1.46
## 10          4.9         3.1          1.5         0.1 setosa      1.46
## # ... with 140 more rows
\end{verbatim}

\begin{Shaded}
\begin{Highlighting}[]
\NormalTok{flores }\OperatorTok
\StringTok{  }\KeywordTok{group_by}\NormalTok{(Species) }\OperatorTok\StringTok{ }
\StringTok{  }\KeywordTok{transmute}\NormalTok{(PL.mé}\DataTypeTok{dia =} \KeywordTok{mean}\NormalTok{(Petal.Length))}
\end{Highlighting}
\end{Shaded}

\begin{verbatim}
## # A tibble: 150 x 2
## # Groups:   Species [3]
##    Species PL.média
##    <fct>      <dbl>
##  1 setosa      1.46
##  2 setosa      1.46
##  3 setosa      1.46
##  4 setosa      1.46
##  5 setosa      1.46
##  6 setosa      1.46
##  7 setosa      1.46
##  8 setosa      1.46
##  9 setosa      1.46
## 10 setosa      1.46
## # ... with 140 more rows
\end{verbatim}

\begin{Shaded}
\begin{Highlighting}[]
\NormalTok{flores }\OperatorTok
\StringTok{  }\KeywordTok{group_by}\NormalTok{(Species) }\OperatorTok\StringTok{ }
\StringTok{  }\KeywordTok{transmute}\NormalTok{(PL.mé}\DataTypeTok{dia =} \KeywordTok{mean}\NormalTok{(Petal.Length), }
\NormalTok{            SL.mé}\DataTypeTok{dia =} \KeywordTok{mean}\NormalTok{(Sepal.Length))}
\end{Highlighting}
\end{Shaded}

\begin{verbatim}
## # A tibble: 150 x 3
## # Groups:   Species [3]
##    Species PL.média SL.média
##    <fct>      <dbl>    <dbl>
##  1 setosa      1.46     5.01
##  2 setosa      1.46     5.01
##  3 setosa      1.46     5.01
##  4 setosa      1.46     5.01
##  5 setosa      1.46     5.01
##  6 setosa      1.46     5.01
##  7 setosa      1.46     5.01
##  8 setosa      1.46     5.01
##  9 setosa      1.46     5.01
## 10 setosa      1.46     5.01
## # ... with 140 more rows
\end{verbatim}

\hypertarget{na_if---substitui-o-valor-especificado-por-na}{%
\section{na\_if() - Substitui o valor especificado por NA}\label{na_if---substitui-o-valor-especificado-por-na}}

\begin{Shaded}
\begin{Highlighting}[]
\NormalTok{flores }\OperatorTok\StringTok{ }
\StringTok{  }\KeywordTok{mutate}\NormalTok{(}\DataTypeTok{Species =} \KeywordTok{na_if}\NormalTok{(Species, }\StringTok{"setosa"}\NormalTok{))}
\end{Highlighting}
\end{Shaded}

\begin{verbatim}
## # A tibble: 150 x 5
##    Sepal.Length Sepal.Width Petal.Length Petal.Width Species
##           <dbl>       <dbl>        <dbl>       <dbl> <fct>  
##  1          5.1         3.5          1.4         0.2 <NA>   
##  2          4.9         3            1.4         0.2 <NA>   
##  3          4.7         3.2          1.3         0.2 <NA>   
##  4          4.6         3.1          1.5         0.2 <NA>   
##  5          5           3.6          1.4         0.2 <NA>   
##  6          5.4         3.9          1.7         0.4 <NA>   
##  7          4.6         3.4          1.4         0.3 <NA>   
##  8          5           3.4          1.5         0.2 <NA>   
##  9          4.4         2.9          1.4         0.2 <NA>   
## 10          4.9         3.1          1.5         0.1 <NA>   
## # ... with 140 more rows
\end{verbatim}

\begin{Shaded}
\begin{Highlighting}[]
\NormalTok{flores }\OperatorTok
\StringTok{  }\KeywordTok{mutate}\NormalTok{(}\DataTypeTok{Petal.Length =} \KeywordTok{na_if}\NormalTok{(Petal.Length, }\FloatTok{1.4}\NormalTok{))}
\end{Highlighting}
\end{Shaded}

\begin{verbatim}
## # A tibble: 150 x 5
##    Sepal.Length Sepal.Width Petal.Length Petal.Width Species
##           <dbl>       <dbl>        <dbl>       <dbl> <fct>  
##  1          5.1         3.5         NA           0.2 setosa 
##  2          4.9         3           NA           0.2 setosa 
##  3          4.7         3.2          1.3         0.2 setosa 
##  4          4.6         3.1          1.5         0.2 setosa 
##  5          5           3.6         NA           0.2 setosa 
##  6          5.4         3.9          1.7         0.4 setosa 
##  7          4.6         3.4         NA           0.3 setosa 
##  8          5           3.4          1.5         0.2 setosa 
##  9          4.4         2.9         NA           0.2 setosa 
## 10          4.9         3.1          1.5         0.1 setosa 
## # ... with 140 more rows
\end{verbatim}

\begin{Shaded}
\begin{Highlighting}[]
\NormalTok{flores }\OperatorTok
\StringTok{  }\KeywordTok{mutate}\NormalTok{(}\DataTypeTok{Sepal.Length =} \KeywordTok{na_if}\NormalTok{(Sepal.Length, }\FloatTok{5.1}\NormalTok{))}
\end{Highlighting}
\end{Shaded}

\begin{verbatim}
## # A tibble: 150 x 5
##    Sepal.Length Sepal.Width Petal.Length Petal.Width Species
##           <dbl>       <dbl>        <dbl>       <dbl> <fct>  
##  1         NA           3.5          1.4         0.2 setosa 
##  2          4.9         3            1.4         0.2 setosa 
##  3          4.7         3.2          1.3         0.2 setosa 
##  4          4.6         3.1          1.5         0.2 setosa 
##  5          5           3.6          1.4         0.2 setosa 
##  6          5.4         3.9          1.7         0.4 setosa 
##  7          4.6         3.4          1.4         0.3 setosa 
##  8          5           3.4          1.5         0.2 setosa 
##  9          4.4         2.9          1.4         0.2 setosa 
## 10          4.9         3.1          1.5         0.1 setosa 
## # ... with 140 more rows
\end{verbatim}

\hypertarget{recode-e-recode_factor---substitui-um-determinado-valor-por-outro-se-variuxe1vel-for-nuxfamerica-usar-recode-se-for-fator-usar-recode_factor}{%
\section{recode() e recode\_factor() - substitui um determinado valor por outro, se variável for númerica usar recode(), se for fator usar recode\_factor()}\label{recode-e-recode_factor---substitui-um-determinado-valor-por-outro-se-variuxe1vel-for-nuxfamerica-usar-recode-se-for-fator-usar-recode_factor}}

\begin{Shaded}
\begin{Highlighting}[]
\NormalTok{flores }\OperatorTok\StringTok{ }
\StringTok{  }\KeywordTok{mutate}\NormalTok{(}\DataTypeTok{Sepal.Length =} \KeywordTok{recode}\NormalTok{(Sepal.Length, }\StringTok{`}\DataTypeTok{5.1}\StringTok{`}\NormalTok{ =}\StringTok{ }\DecValTok{0}\NormalTok{))}
\end{Highlighting}
\end{Shaded}

\begin{verbatim}
## # A tibble: 150 x 5
##    Sepal.Length Sepal.Width Petal.Length Petal.Width Species
##           <dbl>       <dbl>        <dbl>       <dbl> <fct>  
##  1          0           3.5          1.4         0.2 setosa 
##  2          4.9         3            1.4         0.2 setosa 
##  3          4.7         3.2          1.3         0.2 setosa 
##  4          4.6         3.1          1.5         0.2 setosa 
##  5          5           3.6          1.4         0.2 setosa 
##  6          5.4         3.9          1.7         0.4 setosa 
##  7          4.6         3.4          1.4         0.3 setosa 
##  8          5           3.4          1.5         0.2 setosa 
##  9          4.4         2.9          1.4         0.2 setosa 
## 10          4.9         3.1          1.5         0.1 setosa 
## # ... with 140 more rows
\end{verbatim}

\begin{Shaded}
\begin{Highlighting}[]
\NormalTok{flores }\OperatorTok
\StringTok{  }\KeywordTok{mutate}\NormalTok{(}\DataTypeTok{Species =} \KeywordTok{recode_factor}\NormalTok{(Species, }\DataTypeTok{setosa =} \StringTok{"sts"}\NormalTok{))}
\end{Highlighting}
\end{Shaded}

\begin{verbatim}
## # A tibble: 150 x 5
##    Sepal.Length Sepal.Width Petal.Length Petal.Width Species
##           <dbl>       <dbl>        <dbl>       <dbl> <fct>  
##  1          5.1         3.5          1.4         0.2 sts    
##  2          4.9         3            1.4         0.2 sts    
##  3          4.7         3.2          1.3         0.2 sts    
##  4          4.6         3.1          1.5         0.2 sts    
##  5          5           3.6          1.4         0.2 sts    
##  6          5.4         3.9          1.7         0.4 sts    
##  7          4.6         3.4          1.4         0.3 sts    
##  8          5           3.4          1.5         0.2 sts    
##  9          4.4         2.9          1.4         0.2 sts    
## 10          4.9         3.1          1.5         0.1 sts    
## # ... with 140 more rows
\end{verbatim}

\hypertarget{relocate---altera-a-ordem-das-variuxe1veis}{%
\section{relocate() - altera a ordem das variáveis}\label{relocate---altera-a-ordem-das-variuxe1veis}}

\begin{Shaded}
\begin{Highlighting}[]
\NormalTok{flores }\OperatorTok\StringTok{ }
\StringTok{  }\KeywordTok{relocate}\NormalTok{(Species, }\DataTypeTok{.before =}\NormalTok{ Sepal.Length)}
\end{Highlighting}
\end{Shaded}

\begin{verbatim}
## # A tibble: 150 x 5
##    Species Sepal.Length Sepal.Width Petal.Length Petal.Width
##    <fct>          <dbl>       <dbl>        <dbl>       <dbl>
##  1 setosa           5.1         3.5          1.4         0.2
##  2 setosa           4.9         3            1.4         0.2
##  3 setosa           4.7         3.2          1.3         0.2
##  4 setosa           4.6         3.1          1.5         0.2
##  5 setosa           5           3.6          1.4         0.2
##  6 setosa           5.4         3.9          1.7         0.4
##  7 setosa           4.6         3.4          1.4         0.3
##  8 setosa           5           3.4          1.5         0.2
##  9 setosa           4.4         2.9          1.4         0.2
## 10 setosa           4.9         3.1          1.5         0.1
## # ... with 140 more rows
\end{verbatim}

\hypertarget{rename---altera-o-nome-das-variuxe1veis}{%
\section{rename() - altera o nome das variáveis}\label{rename---altera-o-nome-das-variuxe1veis}}

\begin{Shaded}
\begin{Highlighting}[]
\NormalTok{flores }\OperatorTok\StringTok{ }
\StringTok{  }\KeywordTok{rename}\NormalTok{(}\DataTypeTok{sp =}\NormalTok{ Species)}
\end{Highlighting}
\end{Shaded}

\begin{verbatim}
## # A tibble: 150 x 5
##    Sepal.Length Sepal.Width Petal.Length Petal.Width sp    
##           <dbl>       <dbl>        <dbl>       <dbl> <fct> 
##  1          5.1         3.5          1.4         0.2 setosa
##  2          4.9         3            1.4         0.2 setosa
##  3          4.7         3.2          1.3         0.2 setosa
##  4          4.6         3.1          1.5         0.2 setosa
##  5          5           3.6          1.4         0.2 setosa
##  6          5.4         3.9          1.7         0.4 setosa
##  7          4.6         3.4          1.4         0.3 setosa
##  8          5           3.4          1.5         0.2 setosa
##  9          4.4         2.9          1.4         0.2 setosa
## 10          4.9         3.1          1.5         0.1 setosa
## # ... with 140 more rows
\end{verbatim}

\hypertarget{select---seleciona-variuxe1veis}{%
\section{select() - Seleciona variáveis}\label{select---seleciona-variuxe1veis}}

\begin{Shaded}
\begin{Highlighting}[]
\NormalTok{flores }\OperatorTok\StringTok{ }
\StringTok{  }\KeywordTok{select}\NormalTok{(Species)}
\end{Highlighting}
\end{Shaded}

\begin{verbatim}
## # A tibble: 150 x 1
##    Species
##    <fct>  
##  1 setosa 
##  2 setosa 
##  3 setosa 
##  4 setosa 
##  5 setosa 
##  6 setosa 
##  7 setosa 
##  8 setosa 
##  9 setosa 
## 10 setosa 
## # ... with 140 more rows
\end{verbatim}

\begin{Shaded}
\begin{Highlighting}[]
\NormalTok{flores }\OperatorTok\StringTok{ }
\StringTok{  }\KeywordTok{select}\NormalTok{(}\KeywordTok{starts_with}\NormalTok{(}\StringTok{"Sepal"}\NormalTok{))}
\end{Highlighting}
\end{Shaded}

\begin{verbatim}
## # A tibble: 150 x 2
##    Sepal.Length Sepal.Width
##           <dbl>       <dbl>
##  1          5.1         3.5
##  2          4.9         3  
##  3          4.7         3.2
##  4          4.6         3.1
##  5          5           3.6
##  6          5.4         3.9
##  7          4.6         3.4
##  8          5           3.4
##  9          4.4         2.9
## 10          4.9         3.1
## # ... with 140 more rows
\end{verbatim}

\begin{Shaded}
\begin{Highlighting}[]
\NormalTok{flores }\OperatorTok\StringTok{ }
\StringTok{  }\KeywordTok{select}\NormalTok{(}\KeywordTok{ends_with}\NormalTok{(}\StringTok{"Length"}\NormalTok{))}
\end{Highlighting}
\end{Shaded}

\begin{verbatim}
## # A tibble: 150 x 2
##    Sepal.Length Petal.Length
##           <dbl>        <dbl>
##  1          5.1          1.4
##  2          4.9          1.4
##  3          4.7          1.3
##  4          4.6          1.5
##  5          5            1.4
##  6          5.4          1.7
##  7          4.6          1.4
##  8          5            1.5
##  9          4.4          1.4
## 10          4.9          1.5
## # ... with 140 more rows
\end{verbatim}

\begin{Shaded}
\begin{Highlighting}[]
\NormalTok{flores }\OperatorTok\StringTok{ }
\StringTok{    }\KeywordTok{select}\NormalTok{(Species, Sepal.Length)}
\end{Highlighting}
\end{Shaded}

\begin{verbatim}
## # A tibble: 150 x 2
##    Species Sepal.Length
##    <fct>          <dbl>
##  1 setosa           5.1
##  2 setosa           4.9
##  3 setosa           4.7
##  4 setosa           4.6
##  5 setosa           5  
##  6 setosa           5.4
##  7 setosa           4.6
##  8 setosa           5  
##  9 setosa           4.4
## 10 setosa           4.9
## # ... with 140 more rows
\end{verbatim}

\hypertarget{slice---seleciona-linhas}{%
\section{slice() - Seleciona linhas}\label{slice---seleciona-linhas}}

\begin{Shaded}
\begin{Highlighting}[]
\NormalTok{flores }\OperatorTok\StringTok{ }
\StringTok{    }\KeywordTok{slice}\NormalTok{(}\DecValTok{3}\OperatorTok{:}\DecValTok{15}\NormalTok{)}
\end{Highlighting}
\end{Shaded}

\begin{verbatim}
## # A tibble: 13 x 5
##    Sepal.Length Sepal.Width Petal.Length Petal.Width Species
##           <dbl>       <dbl>        <dbl>       <dbl> <fct>  
##  1          4.7         3.2          1.3         0.2 setosa 
##  2          4.6         3.1          1.5         0.2 setosa 
##  3          5           3.6          1.4         0.2 setosa 
##  4          5.4         3.9          1.7         0.4 setosa 
##  5          4.6         3.4          1.4         0.3 setosa 
##  6          5           3.4          1.5         0.2 setosa 
##  7          4.4         2.9          1.4         0.2 setosa 
##  8          4.9         3.1          1.5         0.1 setosa 
##  9          5.4         3.7          1.5         0.2 setosa 
## 10          4.8         3.4          1.6         0.2 setosa 
## 11          4.8         3            1.4         0.1 setosa 
## 12          4.3         3            1.1         0.1 setosa 
## 13          5.8         4            1.2         0.2 setosa
\end{verbatim}

\begin{Shaded}
\begin{Highlighting}[]
\NormalTok{flores }\OperatorTok\StringTok{ }
\StringTok{    }\KeywordTok{slice_sample}\NormalTok{(}\DataTypeTok{n =} \DecValTok{10}\NormalTok{)}
\end{Highlighting}
\end{Shaded}

\begin{verbatim}
## # A tibble: 10 x 5
##    Sepal.Length Sepal.Width Petal.Length Petal.Width Species   
##           <dbl>       <dbl>        <dbl>       <dbl> <fct>     
##  1          5.4         3            4.5         1.5 versicolor
##  2          6.4         2.8          5.6         2.1 virginica 
##  3          5.7         3.8          1.7         0.3 setosa    
##  4          6.7         3.3          5.7         2.1 virginica 
##  5          6.3         3.3          6           2.5 virginica 
##  6          7.7         2.8          6.7         2   virginica 
##  7          5.3         3.7          1.5         0.2 setosa    
##  8          5.6         3            4.1         1.3 versicolor
##  9          6.3         2.9          5.6         1.8 virginica 
## 10          5.8         2.7          5.1         1.9 virginica
\end{verbatim}

\begin{Shaded}
\begin{Highlighting}[]
\NormalTok{flores }\OperatorTok\StringTok{ }
\StringTok{    }\KeywordTok{slice_min}\NormalTok{(Petal.Length, }\DataTypeTok{n =} \DecValTok{10}\NormalTok{)}
\end{Highlighting}
\end{Shaded}

\begin{verbatim}
## # A tibble: 11 x 5
##    Sepal.Length Sepal.Width Petal.Length Petal.Width Species
##           <dbl>       <dbl>        <dbl>       <dbl> <fct>  
##  1          4.6         3.6          1           0.2 setosa 
##  2          4.3         3            1.1         0.1 setosa 
##  3          5.8         4            1.2         0.2 setosa 
##  4          5           3.2          1.2         0.2 setosa 
##  5          4.7         3.2          1.3         0.2 setosa 
##  6          5.4         3.9          1.3         0.4 setosa 
##  7          5.5         3.5          1.3         0.2 setosa 
##  8          4.4         3            1.3         0.2 setosa 
##  9          5           3.5          1.3         0.3 setosa 
## 10          4.5         2.3          1.3         0.3 setosa 
## 11          4.4         3.2          1.3         0.2 setosa
\end{verbatim}

\begin{Shaded}
\begin{Highlighting}[]
\NormalTok{flores }\OperatorTok\StringTok{ }
\StringTok{    }\KeywordTok{slice_max}\NormalTok{(Petal.Length, }\DataTypeTok{n =} \DecValTok{10}\NormalTok{)}
\end{Highlighting}
\end{Shaded}

\begin{verbatim}
## # A tibble: 11 x 5
##    Sepal.Length Sepal.Width Petal.Length Petal.Width Species  
##           <dbl>       <dbl>        <dbl>       <dbl> <fct>    
##  1          7.7         2.6          6.9         2.3 virginica
##  2          7.7         3.8          6.7         2.2 virginica
##  3          7.7         2.8          6.7         2   virginica
##  4          7.6         3            6.6         2.1 virginica
##  5          7.9         3.8          6.4         2   virginica
##  6          7.3         2.9          6.3         1.8 virginica
##  7          7.2         3.6          6.1         2.5 virginica
##  8          7.4         2.8          6.1         1.9 virginica
##  9          7.7         3            6.1         2.3 virginica
## 10          6.3         3.3          6           2.5 virginica
## 11          7.2         3.2          6           1.8 virginica
\end{verbatim}

\hypertarget{summarise---sumariza-os-dados}{%
\section{summarise() - sumariza os dados}\label{summarise---sumariza-os-dados}}

\begin{Shaded}
\begin{Highlighting}[]
\NormalTok{flores }\OperatorTok
\StringTok{  }\KeywordTok{group_by}\NormalTok{(Species) }\OperatorTok\StringTok{ }
\StringTok{  }\KeywordTok{summarise}\NormalTok{(}\DataTypeTok{N =} \KeywordTok{n}\NormalTok{(),}
\NormalTok{            PL.mé}\DataTypeTok{dia =} \KeywordTok{mean}\NormalTok{(Petal.Length),}
\NormalTok{            SL.mé}\DataTypeTok{dia =} \KeywordTok{mean}\NormalTok{(Sepal.Length),}
\NormalTok{            PW.mé}\DataTypeTok{dia =} \KeywordTok{mean}\NormalTok{(Petal.Width),}
\NormalTok{            SW.mé}\DataTypeTok{dia =} \KeywordTok{mean}\NormalTok{(Sepal.Width))}
\end{Highlighting}
\end{Shaded}

\begin{verbatim}
## # A tibble: 3 x 6
##   Species        N PL.média SL.média PW.média SW.média
##   <fct>      <int>    <dbl>    <dbl>    <dbl>    <dbl>
## 1 setosa        50     1.46     5.01    0.246     3.43
## 2 versicolor    50     4.26     5.94    1.33      2.77
## 3 virginica     50     5.55     6.59    2.03      2.97
\end{verbatim}

\hypertarget{tidyr}{%
\chapter{\texorpdfstring{\emph{tidyr}}{tidyr}}\label{tidyr}}

\hypertarget{drop_na---remove-as-linhas-com-na}{%
\section{drop\_na() - remove as linhas com NA}\label{drop_na---remove-as-linhas-com-na}}

\begin{Shaded}
\begin{Highlighting}[]
\NormalTok{starwars }\OperatorTok\StringTok{ }
\StringTok{  }\KeywordTok{select}\NormalTok{(hair_color) }\OperatorTok\StringTok{ }
\StringTok{  }\KeywordTok{drop_na}\NormalTok{()}
\end{Highlighting}
\end{Shaded}

\begin{verbatim}
## # A tibble: 82 x 1
##    hair_color   
##    <chr>        
##  1 blond        
##  2 none         
##  3 brown        
##  4 brown, grey  
##  5 brown        
##  6 black        
##  7 auburn, white
##  8 blond        
##  9 auburn, grey 
## 10 brown        
## # ... with 72 more rows
\end{verbatim}

\begin{Shaded}
\begin{Highlighting}[]
\NormalTok{starwars }\OperatorTok
\StringTok{  }\KeywordTok{drop_na}\NormalTok{()}
\end{Highlighting}
\end{Shaded}

\begin{verbatim}
## # A tibble: 6 x 14
##   name     height  mass hair_color  skin_color eye_color birth_year sex   gender
##   <chr>     <int> <dbl> <chr>       <chr>      <chr>          <dbl> <chr> <chr> 
## 1 Luke Sk~    172    77 blond       fair       blue            19   male  mascu~
## 2 Obi-Wan~    182    77 auburn, wh~ fair       blue-gray       57   male  mascu~
## 3 Anakin ~    188    84 blond       fair       blue            41.9 male  mascu~
## 4 Chewbac~    228   112 brown       unknown    blue           200   male  mascu~
## 5 Wedge A~    170    77 brown       fair       hazel           21   male  mascu~
## 6 Darth M~    175    80 none        red        yellow          54   male  mascu~
## # ... with 5 more variables: homeworld <chr>, species <chr>, films <list>,
## #   vehicles <list>, starships <list>
\end{verbatim}

\hypertarget{replace_na---substitui-os-valores-de-na-por-outro-valor}{%
\section{replace\_na() - Substitui os valores de NA por outro valor}\label{replace_na---substitui-os-valores-de-na-por-outro-valor}}

\begin{Shaded}
\begin{Highlighting}[]
\NormalTok{starwars }\OperatorTok\StringTok{ }
\StringTok{  }\KeywordTok{replace_na}\NormalTok{(}\KeywordTok{list}\NormalTok{(}\DataTypeTok{hair_color =} \StringTok{"orange"}\NormalTok{))}
\end{Highlighting}
\end{Shaded}

\begin{verbatim}
## # A tibble: 87 x 14
##    name    height  mass hair_color  skin_color eye_color birth_year sex   gender
##    <chr>    <int> <dbl> <chr>       <chr>      <chr>          <dbl> <chr> <chr> 
##  1 Luke S~    172    77 blond       fair       blue            19   male  mascu~
##  2 C-3PO      167    75 orange      gold       yellow         112   none  mascu~
##  3 R2-D2       96    32 orange      white, bl~ red             33   none  mascu~
##  4 Darth ~    202   136 none        white      yellow          41.9 male  mascu~
##  5 Leia O~    150    49 brown       light      brown           19   fema~ femin~
##  6 Owen L~    178   120 brown, grey light      blue            52   male  mascu~
##  7 Beru W~    165    75 brown       light      blue            47   fema~ femin~
##  8 R5-D4       97    32 orange      white, red red             NA   none  mascu~
##  9 Biggs ~    183    84 black       light      brown           24   male  mascu~
## 10 Obi-Wa~    182    77 auburn, wh~ fair       blue-gray       57   male  mascu~
## # ... with 77 more rows, and 5 more variables: homeworld <chr>, species <chr>,
## #   films <list>, vehicles <list>, starships <list>
\end{verbatim}

\hypertarget{pivot_longer-e-pivot_wider---aumenta-o-nuxfamero-de-linhas-e-diminui-o-nuxfamero-de-colunas-aumenta-o-nuxfamero-de-colunas-e-diminui-o-nuxfamero-de-linhas}{%
\section{pivot\_longer() e pivot\_wider() - Aumenta o número de linhas e diminui o número de colunas; aumenta o número de colunas e diminui o número de linhas}\label{pivot_longer-e-pivot_wider---aumenta-o-nuxfamero-de-linhas-e-diminui-o-nuxfamero-de-colunas-aumenta-o-nuxfamero-de-colunas-e-diminui-o-nuxfamero-de-linhas}}

\begin{Shaded}
\begin{Highlighting}[]
\NormalTok{starwars[}\DecValTok{1}\OperatorTok{:}\DecValTok{10}\NormalTok{,] }\OperatorTok
\StringTok{  }\KeywordTok{select}\NormalTok{(homeworld, skin_color, mass) }\OperatorTok\StringTok{ }
\StringTok{  }\KeywordTok{pivot_wider}\NormalTok{(}\DataTypeTok{names_from =}\NormalTok{ homeworld, }\DataTypeTok{values_from =}\NormalTok{ mass, }\DataTypeTok{values_fn =}\NormalTok{ list)}
\end{Highlighting}
\end{Shaded}

\begin{verbatim}
## # A tibble: 6 x 5
##   skin_color  Tatooine  Naboo     Alderaan  Stewjon  
##   <chr>       <list>    <list>    <list>    <list>   
## 1 fair        <dbl [1]> <NULL>    <NULL>    <dbl [1]>
## 2 gold        <dbl [1]> <NULL>    <NULL>    <NULL>   
## 3 white, blue <NULL>    <dbl [1]> <NULL>    <NULL>   
## 4 white       <dbl [1]> <NULL>    <NULL>    <NULL>   
## 5 light       <dbl [3]> <NULL>    <dbl [1]> <NULL>   
## 6 white, red  <dbl [1]> <NULL>    <NULL>    <NULL>
\end{verbatim}

\begin{Shaded}
\begin{Highlighting}[]
\NormalTok{starwars[}\DecValTok{1}\OperatorTok{:}\DecValTok{10}\NormalTok{,] }\OperatorTok
\StringTok{  }\KeywordTok{select}\NormalTok{(homeworld, skin_color, mass) }\OperatorTok\StringTok{ }
\StringTok{  }\KeywordTok{pivot_wider}\NormalTok{(}\DataTypeTok{names_from =}\NormalTok{ homeworld, }\DataTypeTok{values_from =}\NormalTok{ mass, }\DataTypeTok{values_fn =}\NormalTok{ list) }\OperatorTok\StringTok{ }
\StringTok{  }\KeywordTok{view}\NormalTok{()}

\NormalTok{starwars[}\DecValTok{1}\OperatorTok{:}\DecValTok{10}\NormalTok{,] }\OperatorTok
\StringTok{  }\KeywordTok{select}\NormalTok{(homeworld, skin_color, mass) }\OperatorTok\StringTok{ }
\StringTok{  }\KeywordTok{pivot_wider}\NormalTok{(}\DataTypeTok{names_from =}\NormalTok{ homeworld, }\DataTypeTok{values_from =}\NormalTok{ mass) }\OperatorTok\StringTok{ }
\StringTok{  }\KeywordTok{unchop}\NormalTok{(}\KeywordTok{c}\NormalTok{(}\DecValTok{2}\OperatorTok{:}\DecValTok{5}\NormalTok{))}
\end{Highlighting}
\end{Shaded}

\begin{verbatim}
## Warning: Values are not uniquely identified; output will contain list-cols.
## * Use `values_fn = list` to suppress this warning.
## * Use `values_fn = length` to identify where the duplicates arise
## * Use `values_fn = {summary_fun}` to summarise duplicates
\end{verbatim}

\begin{verbatim}
## # A tibble: 8 x 5
##   skin_color  Tatooine Naboo Alderaan Stewjon
##   <chr>          <dbl> <dbl>    <dbl>   <dbl>
## 1 fair              77    NA       NA      77
## 2 gold              75    NA       NA      NA
## 3 white, blue       NA    32       NA      NA
## 4 white            136    NA       NA      NA
## 5 light            120    NA       49      NA
## 6 light             75    NA       49      NA
## 7 light             84    NA       49      NA
## 8 white, red        32    NA       NA      NA
\end{verbatim}

\begin{Shaded}
\begin{Highlighting}[]
\NormalTok{starwars[}\DecValTok{1}\OperatorTok{:}\DecValTok{10}\NormalTok{,] }\OperatorTok
\StringTok{  }\KeywordTok{select}\NormalTok{(skin_color, homeworld, mass) }\OperatorTok
\StringTok{  }\KeywordTok{pivot_wider}\NormalTok{(}\DataTypeTok{names_from =}\NormalTok{ homeworld, }\DataTypeTok{values_from =}\NormalTok{ mass) }\OperatorTok
\StringTok{  }\KeywordTok{pivot_longer}\NormalTok{(}\DataTypeTok{cols =} \DecValTok{2}\OperatorTok{:}\DecValTok{5}\NormalTok{, }\DataTypeTok{names_to =} \StringTok{"homerworld"}\NormalTok{, }\DataTypeTok{values_to =} \StringTok{"mass"}\NormalTok{)}
\end{Highlighting}
\end{Shaded}

\begin{verbatim}
## Warning: Values are not uniquely identified; output will contain list-cols.
## * Use `values_fn = list` to suppress this warning.
## * Use `values_fn = length` to identify where the duplicates arise
## * Use `values_fn = {summary_fun}` to summarise duplicates
\end{verbatim}

\begin{verbatim}
## # A tibble: 24 x 3
##    skin_color  homerworld mass     
##    <chr>       <chr>      <list>   
##  1 fair        Tatooine   <dbl [1]>
##  2 fair        Naboo      <NULL>   
##  3 fair        Alderaan   <NULL>   
##  4 fair        Stewjon    <dbl [1]>
##  5 gold        Tatooine   <dbl [1]>
##  6 gold        Naboo      <NULL>   
##  7 gold        Alderaan   <NULL>   
##  8 gold        Stewjon    <NULL>   
##  9 white, blue Tatooine   <NULL>   
## 10 white, blue Naboo      <dbl [1]>
## # ... with 14 more rows
\end{verbatim}

\begin{Shaded}
\begin{Highlighting}[]
\NormalTok{starwars[}\DecValTok{1}\OperatorTok{:}\DecValTok{10}\NormalTok{,] }\OperatorTok
\StringTok{  }\KeywordTok{select}\NormalTok{(homeworld, skin_color, mass) }\OperatorTok
\StringTok{  }\KeywordTok{pivot_wider}\NormalTok{(}\DataTypeTok{names_from =}\NormalTok{ homeworld, }\DataTypeTok{values_from =}\NormalTok{ mass) }\OperatorTok
\StringTok{  }\KeywordTok{pivot_longer}\NormalTok{(}\DataTypeTok{cols =} \DecValTok{2}\OperatorTok{:}\DecValTok{5}\NormalTok{, }\DataTypeTok{names_to =} \StringTok{"homeworld"}\NormalTok{, }\DataTypeTok{values_to =} \StringTok{"mass"}\NormalTok{) }\OperatorTok
\StringTok{  }\KeywordTok{unchop}\NormalTok{(}\KeywordTok{everything}\NormalTok{()) }\OperatorTok\StringTok{ }
\StringTok{  }\KeywordTok{drop_na}\NormalTok{()}
\end{Highlighting}
\end{Shaded}

\begin{verbatim}
## Warning: Values are not uniquely identified; output will contain list-cols.
## * Use `values_fn = list` to suppress this warning.
## * Use `values_fn = length` to identify where the duplicates arise
## * Use `values_fn = {summary_fun}` to summarise duplicates
\end{verbatim}

\begin{verbatim}
## # A tibble: 10 x 3
##    skin_color  homeworld  mass
##    <chr>       <chr>     <dbl>
##  1 fair        Tatooine     77
##  2 fair        Stewjon      77
##  3 gold        Tatooine     75
##  4 white, blue Naboo        32
##  5 white       Tatooine    136
##  6 light       Tatooine    120
##  7 light       Tatooine     75
##  8 light       Tatooine     84
##  9 light       Alderaan     49
## 10 white, red  Tatooine     32
\end{verbatim}

\hypertarget{separate-e-unite---separa-uma-coluna-em-muxfaltiplas-colunas-une-muxfaltiplas-colunas}{%
\section{separate() e unite() - Separa uma coluna em múltiplas colunas; Une múltiplas colunas}\label{separate-e-unite---separa-uma-coluna-em-muxfaltiplas-colunas-une-muxfaltiplas-colunas}}

\begin{Shaded}
\begin{Highlighting}[]
\NormalTok{starwars[}\DecValTok{1}\OperatorTok{:}\DecValTok{10}\NormalTok{,] }\OperatorTok\StringTok{ }
\StringTok{  }\KeywordTok{select}\NormalTok{(sex, gender, homeworld) }\OperatorTok\StringTok{ }
\StringTok{  }\KeywordTok{unite}\NormalTok{(}\StringTok{"sexgender"}\NormalTok{, sex}\OperatorTok{:}\NormalTok{gender, }\DataTypeTok{sep =} \StringTok{"-"}\NormalTok{)}
\end{Highlighting}
\end{Shaded}

\begin{verbatim}
## # A tibble: 10 x 2
##    sexgender       homeworld
##    <chr>           <chr>    
##  1 male-masculine  Tatooine 
##  2 none-masculine  Tatooine 
##  3 none-masculine  Naboo    
##  4 male-masculine  Tatooine 
##  5 female-feminine Alderaan 
##  6 male-masculine  Tatooine 
##  7 female-feminine Tatooine 
##  8 none-masculine  Tatooine 
##  9 male-masculine  Tatooine 
## 10 male-masculine  Stewjon
\end{verbatim}

\begin{Shaded}
\begin{Highlighting}[]
\NormalTok{starwars[}\DecValTok{1}\OperatorTok{:}\DecValTok{10}\NormalTok{,] }\OperatorTok\StringTok{ }
\StringTok{  }\KeywordTok{select}\NormalTok{(sex, gender, homeworld) }\OperatorTok\StringTok{ }
\StringTok{  }\KeywordTok{unite}\NormalTok{(}\StringTok{"sexgender"}\NormalTok{, sex}\OperatorTok{:}\NormalTok{gender, }\DataTypeTok{sep =} \StringTok{"-"}\NormalTok{) }\OperatorTok\StringTok{ }
\StringTok{  }\KeywordTok{separate}\NormalTok{(sexgender, }\KeywordTok{c}\NormalTok{(}\StringTok{"sex"}\NormalTok{, }\StringTok{"gender"}\NormalTok{), }\DataTypeTok{sep =} \StringTok{"-"}\NormalTok{)}
\end{Highlighting}
\end{Shaded}

\begin{verbatim}
## # A tibble: 10 x 3
##    sex    gender    homeworld
##    <chr>  <chr>     <chr>    
##  1 male   masculine Tatooine 
##  2 none   masculine Tatooine 
##  3 none   masculine Naboo    
##  4 male   masculine Tatooine 
##  5 female feminine  Alderaan 
##  6 male   masculine Tatooine 
##  7 female feminine  Tatooine 
##  8 none   masculine Tatooine 
##  9 male   masculine Tatooine 
## 10 male   masculine Stewjon
\end{verbatim}

\hypertarget{fill---preenche-as-cuxe9lulas-com-na-com-o-valor-posterior-ou-anterior-da-mesma-coluna}{%
\section{fill() - Preenche as células com NA com o valor posterior ou anterior da mesma coluna}\label{fill---preenche-as-cuxe9lulas-com-na-com-o-valor-posterior-ou-anterior-da-mesma-coluna}}

\begin{Shaded}
\begin{Highlighting}[]
\NormalTok{starwars }\OperatorTok\StringTok{ }
\StringTok{  }\KeywordTok{select}\NormalTok{(hair_color) }\OperatorTok\StringTok{ }
\StringTok{  }\KeywordTok{fill}\NormalTok{(hair_color)}
\end{Highlighting}
\end{Shaded}

\begin{verbatim}
## # A tibble: 87 x 1
##    hair_color   
##    <chr>        
##  1 blond        
##  2 blond        
##  3 blond        
##  4 none         
##  5 brown        
##  6 brown, grey  
##  7 brown        
##  8 brown        
##  9 black        
## 10 auburn, white
## # ... with 77 more rows
\end{verbatim}

\hypertarget{integrando-os-pacotes-tibble-dplyr-tidyr-e-magrittr}{%
\chapter{\texorpdfstring{Integrando os pacotes \emph{tibble}, \emph{dplyr}, \emph{tidyr} e \emph{magrittr}}{Integrando os pacotes tibble, dplyr, tidyr e magrittr}}\label{integrando-os-pacotes-tibble-dplyr-tidyr-e-magrittr}}

Importar o arquivo excel de nome ``dados'' presente na página para resolução desta tarefa.

\begin{Shaded}
\begin{Highlighting}[]
\KeywordTok{library}\NormalTok{(readxl)}
\NormalTok{dados <-}\StringTok{ }\KeywordTok{read_excel}\NormalTok{(}\StringTok{"dados.xlsx"}\NormalTok{)}

\NormalTok{dados }\OperatorTok\StringTok{ }
\StringTok{  }\KeywordTok{select}\NormalTok{(}\DecValTok{1}\OperatorTok{:}\DecValTok{4}\NormalTok{) }\OperatorTok\StringTok{ }
\StringTok{  }\KeywordTok{distinct}\NormalTok{() }\OperatorTok\StringTok{ }
\StringTok{  }\KeywordTok{filter}\NormalTok{(ano }\OperatorTok{==}\StringTok{ "2020"}\NormalTok{) }\OperatorTok\StringTok{ }
\StringTok{  }\KeywordTok{pivot_wider}\NormalTok{(}\DataTypeTok{id_cols =} \KeywordTok{c}\NormalTok{(ano, mes), }
              \DataTypeTok{names_from =}\NormalTok{ especie, }
              \DataTypeTok{values_from =}\NormalTok{ abundancia)}
\end{Highlighting}
\end{Shaded}

\begin{verbatim}
## # A tibble: 12 x 5
##      ano mes         sp1   sp2   sp3
##    <dbl> <chr>     <dbl> <dbl> <dbl>
##  1  2020 janeiro     132   105    80
##  2  2020 fevereiro    56     3     2
##  3  2020 março        41     4     8
##  4  2020 abril         5    85   166
##  5  2020 maio         72   152    35
##  6  2020 junho        15    38   148
##  7  2020 julho        57   184   141
##  8  2020 agosto       74    NA    55
##  9  2020 setembro     20    55   131
## 10  2020 outubro     184    66    49
## 11  2020 novembro    145   135    72
## 12  2020 dezembro     NA   151   177
\end{verbatim}

\begin{Shaded}
\begin{Highlighting}[]
\NormalTok{dados }\OperatorTok\StringTok{ }
\StringTok{  }\KeywordTok{select}\NormalTok{(mes, temperatura) }\OperatorTok\StringTok{ }
\StringTok{  }\KeywordTok{group_by}\NormalTok{(mes) }\OperatorTok\StringTok{ }
\StringTok{  }\KeywordTok{summarise}\NormalTok{(}\DataTypeTok{temp_media =} \KeywordTok{mean}\NormalTok{(temperatura, }\DataTypeTok{na.rm =} \OtherTok{TRUE}\NormalTok{)) }\OperatorTok\StringTok{ }
\StringTok{  }\KeywordTok{arrange}\NormalTok{(}\KeywordTok{factor}\NormalTok{(mes, }\DataTypeTok{levels =} \KeywordTok{c}\NormalTok{(}\StringTok{"janeiro"}\NormalTok{, }\StringTok{"fevereiro"}\NormalTok{, }\StringTok{"março"}\NormalTok{, }\StringTok{"abril"}\NormalTok{, }\StringTok{"maio"}\NormalTok{, }\StringTok{"junho"}\NormalTok{, }\StringTok{"julho"}\NormalTok{, }\StringTok{"agosto"}\NormalTok{, }\StringTok{"setembro"}\NormalTok{, }\StringTok{"outubro"}\NormalTok{, }\StringTok{"novembro"}\NormalTok{, }\StringTok{"dezembro"}\NormalTok{)))}
\end{Highlighting}
\end{Shaded}

\begin{verbatim}
## # A tibble: 12 x 2
##    mes       temp_media
##    <chr>          <dbl>
##  1 janeiro         19.5
##  2 fevereiro       24.1
##  3 março           21.9
##  4 abril           19.2
##  5 maio            22.1
##  6 junho           22.4
##  7 julho           25.0
##  8 agosto          20.1
##  9 setembro        25.0
## 10 outubro         21.2
## 11 novembro        23.2
## 12 dezembro        22.8
\end{verbatim}

\begin{Shaded}
\begin{Highlighting}[]
\NormalTok{dados }\OperatorTok\StringTok{ }
\StringTok{  }\KeywordTok{select}\NormalTok{(mes, temperatura) }\OperatorTok\StringTok{ }
\StringTok{  }\KeywordTok{group_by}\NormalTok{(mes) }\OperatorTok\StringTok{ }
\StringTok{  }\KeywordTok{summarise}\NormalTok{(}\DataTypeTok{temp_media =} \KeywordTok{mean}\NormalTok{(temperatura, }\DataTypeTok{na.rm =} \OtherTok{TRUE}\NormalTok{), }
            \DataTypeTok{temp_sd =} \KeywordTok{sd}\NormalTok{(temperatura, }\DataTypeTok{na.rm =} \OtherTok{TRUE}\NormalTok{)) }\OperatorTok\StringTok{ }
\StringTok{  }\KeywordTok{arrange}\NormalTok{(}\KeywordTok{factor}\NormalTok{(mes, }\DataTypeTok{levels =} \KeywordTok{c}\NormalTok{(}\StringTok{"janeiro"}\NormalTok{, }\StringTok{"fevereiro"}\NormalTok{, }\StringTok{"março"}\NormalTok{, }\StringTok{"abril"}\NormalTok{, }\StringTok{"maio"}\NormalTok{, }\StringTok{"junho"}\NormalTok{, }\StringTok{"julho"}\NormalTok{, }\StringTok{"agosto"}\NormalTok{, }\StringTok{"setembro"}\NormalTok{, }\StringTok{"outubro"}\NormalTok{, }\StringTok{"novembro"}\NormalTok{, }\StringTok{"dezembro"}\NormalTok{)))}
\end{Highlighting}
\end{Shaded}

\begin{verbatim}
## # A tibble: 12 x 3
##    mes       temp_media temp_sd
##    <chr>          <dbl>   <dbl>
##  1 janeiro         19.5   0.249
##  2 fevereiro       24.1   1.11 
##  3 março           21.9   3.87 
##  4 abril           19.2   0    
##  5 maio            22.1   4.16 
##  6 junho           22.4   4.19 
##  7 julho           25.0   1.16 
##  8 agosto          20.1   2.01 
##  9 setembro        25.0   1.30 
## 10 outubro         21.2   0.954
## 11 novembro        23.2   0.738
## 12 dezembro        22.8   4.00
\end{verbatim}

\begin{Shaded}
\begin{Highlighting}[]
\NormalTok{dados }\OperatorTok\StringTok{ }
\StringTok{  }\KeywordTok{select}\NormalTok{(mes, temperatura, salinidade) }\OperatorTok\StringTok{ }
\StringTok{  }\KeywordTok{group_by}\NormalTok{(mes) }\OperatorTok\StringTok{ }
\StringTok{  }\KeywordTok{summarise}\NormalTok{(}\DataTypeTok{temp_media =} \KeywordTok{mean}\NormalTok{(temperatura, }\DataTypeTok{na.rm =} \OtherTok{TRUE}\NormalTok{), }
            \DataTypeTok{temp_sd =} \KeywordTok{sd}\NormalTok{(temperatura, }\DataTypeTok{na.rm =} \OtherTok{TRUE}\NormalTok{), }
            \DataTypeTok{sal_media =} \KeywordTok{mean}\NormalTok{(salinidade, }\DataTypeTok{na.rm =} \OtherTok{TRUE}\NormalTok{), }
            \DataTypeTok{sal_sd =} \KeywordTok{sd}\NormalTok{(salinidade, }\DataTypeTok{na.rm =} \OtherTok{TRUE}\NormalTok{)) }\OperatorTok\StringTok{ }
\StringTok{  }\KeywordTok{arrange}\NormalTok{(}\KeywordTok{factor}\NormalTok{(mes, }\DataTypeTok{levels =} \KeywordTok{c}\NormalTok{(}\StringTok{"janeiro"}\NormalTok{, }\StringTok{"fevereiro"}\NormalTok{, }\StringTok{"março"}\NormalTok{, }\StringTok{"abril"}\NormalTok{, }\StringTok{"maio"}\NormalTok{, }\StringTok{"junho"}\NormalTok{, }\StringTok{"julho"}\NormalTok{, }\StringTok{"agosto"}\NormalTok{, }\StringTok{"setembro"}\NormalTok{, }\StringTok{"outubro"}\NormalTok{, }\StringTok{"novembro"}\NormalTok{, }\StringTok{"dezembro"}\NormalTok{))) }\OperatorTok
\StringTok{  }\KeywordTok{mutate}\NormalTok{(}\DataTypeTok{temp_sal =}\NormalTok{ temp_media}\OperatorTok{/}\NormalTok{sal_media)}
\end{Highlighting}
\end{Shaded}

\begin{verbatim}
## # A tibble: 12 x 6
##    mes       temp_media temp_sd sal_media  sal_sd temp_sal
##    <chr>          <dbl>   <dbl>     <dbl>   <dbl>    <dbl>
##  1 janeiro         19.5   0.249      24.5 5.30       0.797
##  2 fevereiro       24.1   1.11       26.6 0.473      0.908
##  3 março           21.9   3.87       29.9 2.47       0.732
##  4 abril           19.2   0          17.3 0.00594    1.11 
##  5 maio            22.1   4.16       25.7 7.95       0.861
##  6 junho           22.4   4.19       20.8 3.91       1.08 
##  7 julho           25.0   1.16       28.8 2.19       0.867
##  8 agosto          20.1   2.01       24.0 1.82       0.840
##  9 setembro        25.0   1.30       30.8 0.611      0.810
## 10 outubro         21.2   0.954      22.2 6.51       0.956
## 11 novembro        23.2   0.738      26.1 9.69       0.890
## 12 dezembro        22.8   4.00       18.1 0.00526    1.26
\end{verbatim}

\hypertarget{exercuxedcio}{%
\chapter{Exercício}\label{exercuxedcio}}

Importar o arquivo excel de nome ``tarefa'' presente na página para resolução destes exercícios.

1 - Faça a conversão da planilha para as tabelas a seguir utilizando as funções acima. Tente executar com apenas um comando e o mínimo de funções possíveis. \textbf{Utilize o operador pipe: \%\textgreater{}\%"}. \textbf{Se houver observações iguais remova-as}

2 - Converta as seguintes tabelas (salinidade, temperatura, pH e mortalidade) para a seguinte planilha. Tente executar com mínimo de funções possíveis. \textbf{Dica: converta cada tabela em uma planilha e as una como colunas}. \textbf{``Utilize o operador pipe: \%\textgreater{}\%''}. \textbf{Se houver observações iguais remova-as}

\hypertarget{resposta-exercuxedcio}{%
\chapter{Resposta exercício}\label{resposta-exercuxedcio}}

Será postada assim que todos que participaram da aula enviarem as suas respostas. ;)

\end{document}
